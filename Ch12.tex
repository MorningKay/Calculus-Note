\setcounter{chapter}{11}

\chapter{常微分方程}

\begin{introduction}
    \item 可分离变量的微分方程
    \item 齐次微分方程
    \item 一阶线性微分方程
    \item 二阶常系数微分方程
\end{introduction}

\section{微分方程的基本概念}
含有未知函数的导数或微分的方程称为微分方程

微分方程分两类:常微分方程和偏微分方程。若未知函数为多元函数,微分方程中出现偏导数,这样的微分方程成为偏微分方程。而未知函数为一元函数的微分方程成为常微分方程,本章只限于研究常微分方程,简称微分方程,有时候也简称为方程。

微分方程中未知函数的最高阶导数的阶数称为这个方程的阶。

$n$阶常微分方程的一般形式为:
\begin{equation*}
    F(x,y,y',y'',\cdots,y^{(n)})=0
\end{equation*}

若将函数$y=y(x)$代入微分方程后,能使方程成为恒等式,则称函数$y=y(x)$为微分方程的解。

若微分方程的解中所含独立任意常数的与此微分方程的阶数相等,则称这个解为微分方程的通解。

确定通解中任意常数的条件称为定解条件。

满足定解条件的解成为微分方程的特解。

\section{可分离变量的微分方程}
形如
\begin{equation*}
    f_1(x)g_1(y)\deriv x+f_2(x)g_2(y)\deriv y=0
\end{equation*}
的一阶微分方程称为可分离变量的微分方程。将方程两端除以$g_1(y)f_2(x)$(此时$g_1(y)f_2(x)\neq0$),得
\begin{equation*}
    \dfrac{f_1(x)}{f_2(x)}\deriv x+\dfrac{g_1(y)}{g_2(y)}\deriv y=0
\end{equation*}
然后对上式两端积分,即可得方程的通解
\begin{equation*}
    \int\dfrac{f_1(x)}{f_2(x)}\deriv x+\int\dfrac{g_1(y)}{g_2(y)}\deriv y=C
\end{equation*}

\section{齐次微分方程}
\textbf{1.齐次方程}

形如
\begin{equation*}
    \dfrac{\deriv y}{\deriv x}=f\left(\dfrac{y}{x}\right)
\end{equation*}
的一阶微分方程称为齐次方程.

令$u=\dfrac{y}{x}$,或$y=x\cdot u$,则$\dfrac{\deriv y}{\deriv x}=u+x\dfrac{\deriv u}{\deriv x}$,代入原方程得
\begin{equation*}
    u+x\dfrac{\deriv u}{\deriv x}=f(u)
\end{equation*}
即
\begin{equation*}
    \dfrac{\deriv u}{f(u)-u}=\dfrac{1}{x}\deriv x
\end{equation*}
这是变量已分离的微分方程,经积分即可得方程的通解.

\textbf{2.可化为齐次方程的微分方程}

形如方程
\begin{equation*}
    \dfrac{\deriv y}{\deriv x}=f\left(\dfrac{a_1x+b_1y+c_1}{a_2x+b_2y+c_2}\right)
\end{equation*}
其中$a_1,b_1,c_1,a_2,b_2,c_2$为常数,且$c_1^2+c_2^2\neq0$。当$\begin{vmatrix}
    a_1 & b_1 \\
    a_2 & b_2
\end{vmatrix}\neq0$时,令$x=X+h,y=Y+k$,由
\begin{equation*}
    \begin{cases}
        a_1h+b_1k+c_1=0 \\
        a_2h+b_2k+c_2=0
    \end{cases}
\end{equation*}
解出$h$与$k$,可将原方程化为齐次方程
\begin{equation*}
    \dfrac{\deriv Y}{\deriv X}=f\left(\dfrac{a_1X+b_1Y}{a_2X+b_2Y}\right)=f\left[\dfrac{a_1+b_1\dfrac{Y}{X}}{a_2+b_2\dfrac{Y}{X}}\right]=g\left(\dfrac{Y}{X}\right)
\end{equation*}
当$\begin{vmatrix}
    a_1 & b_1 \\
    a_2 & b_2
\end{vmatrix}=0$时,即$\dfrac{a_1}{a_2}=\dfrac{b_1}{b_2}=k$,可设$u=a_2x+b_2y$,代入原方程后可化为可分离变量的微分方程,即有
\begin{equation*}
    \dfrac{\deriv y}{\deriv x}=f\left(\dfrac{ku+c_1}{u+c_2}\right)=g(u), \quad \dfrac{\deriv u}{\deriv x}=a_2+b_2g(u)
\end{equation*}

\section{一阶线性微分方程}
\textbf{1.一阶线性微分方程}

形如
\begin{equation*}
    \dfrac{\deriv y}{\deriv x}+P(x)y=Q(x)
\end{equation*}
的一阶微分方程称为一阶线性微分方程.

\vspace{2mm}
(1)$\dfrac{\deriv y}{\deriv x}+P(x)y=0$称为一阶齐次线性微分方程,直接积分,可得其通解
\begin{equation*}
    y=Ce^{-\int P(x)\deriv x}
\end{equation*}

(2)$\dfrac{\deriv y}{\deriv x}+P(x)y=Q(x)$称为一阶非齐次线性微分方程,用常数变易法,可得其通解
\begin{equation*}
    y=e^{-\int P(x)\deriv x}\left[\int Q(x)e^{\int P(x)\deriv x}\deriv x+C\right]
\end{equation*}

\begin{note}
    这里补充一个求解一阶非齐次线性微分方程的方法:

    我们现在要求方程
    \begin{equation}
        \dfrac{\deriv y}{\deriv x}+P(x)y=Q(x) \label{linear_equation}
    \end{equation}

    设一个函数(积分因子)$I(x)$,将$I(x)$乘到公式(\ref{linear_equation})的左边,以便于将其凑成$I(x)y$的导数的形式
    \begin{equation}
        I(x)(y'+P(x)y)=(I(x)y)' \label{integrating_factor}
    \end{equation}

    如果我们可以找到这样的一个函数$I$,那么公式(\ref{linear_equation})就可以变为
    \begin{equation*}
        (I(x)y)'=I(x)Q(x)
    \end{equation*}

    然后两边对$x$积分,即可得到
    \begin{equation*}
        I(x)y=\int I(x)Q(x)\deriv x+C
    \end{equation*}

    由此我们可得到解为
    \begin{equation}
        y(x)=\dfrac{1}{I(x)}\left[\int I(x)Q(x)\deriv x+C\right] \label{solution}
    \end{equation}

    为了去找到这样的一个函数$I$,我们再对公式(\ref{integrating_factor})变形
    \begin{align*}
        I(x)y'+I(x)P(x)y=(I(x)y)'=I'(x)y+I(x)y' \\
        I(x)P(x)=I'(x)
    \end{align*}

    把$I$分离出来,可得到一个新的微分方程,然后我们可解得
    \begin{align*}
        \int\dfrac{\deriv I}{I}=\int P(x)\deriv x \\
        \ln |I|=\int P(x)\deriv x \\
        I=Ae^{\int P(x)\deriv x}
    \end{align*}

    其中$A=\pm e^C$是一个常数,我们只需要寻找一个特定的积分因子,不需要找到通解。所以我们令$A=1$,得到
    \begin{equation*}
        I(x)=e^{\int P(x)\deriv x}
    \end{equation*}

    将$I$代入公式(\ref{solution}),即可得到一阶非齐次线性微分方程的通解
\end{note}

\textbf{2.贝努里(Bernoulli)方程}

一阶微分方程$\dfrac{\deriv y}{\deriv x}+P(x)y=Q(x)y^n(n\neq0,1)$称为贝努里方程,用变量代换$z=y^{1-n}$,可化为$z$的一阶线性方程
\begin{equation*}
    \dfrac{\deriv z}{\deriv x}+(1-n)P(x)z=(1-n)Q(x)
\end{equation*}

\section{全微分方程}
\textbf{1.全微分方程}

若方程
\begin{equation}
    P(x,y)\deriv x+Q(x,y)\deriv y=0 \label{total_equation}
\end{equation}
的左端恰好是某一个二元函数$u(x,y)$的全微分,即:
\begin{equation*}
    \deriv u=P(x,y)\deriv x+Q(x,y)\deriv y=\dfrac{\partial u}{\partial x}\deriv x+\dfrac{\partial u}{\partial y}\deriv y
\end{equation*}
则称方程(\ref{total_equation})为全微分方程(或称为恰当方程),全微分方程通解是$u(x,y)=C$($C$是任意常数),$u(x,y)$也称为$P\deriv x+Q\deriv y$的原函数.

\textbf{2.方程(\ref{total_equation})为全微分方程的充要条件及原函数$u(x,y)$的求法}

若$P(x,y),Q(x,y)$在某一单连通域上连续,且有连续的一阶偏导数,则方程(\ref{total_equation})为全微分方程的充要条件是$\dfrac{\partial P}{\partial y}=\dfrac{\partial Q}{\partial x}$,这时有
\begin{equation*}
    u(x,y)=\int_{x_0}^x P(x,y_0)\deriv x+\int_{y_0}^y Q(x,y)\deriv y+C
\end{equation*}
或
\begin{equation*}
    u(x,y)=\int_{x_0}^x P(x,y)\deriv x+\int_{y_0}^y Q(x_0,y)\deriv y+C
\end{equation*}

\textbf{3.积分因子}

若方程(\ref{total_equation})不是全微分方程,但存在一个函数$\mu(x,y)$使
\begin{equation*}
    \mu(x,y)P(x,y)\deriv x+\mu(x,y)Q(x,y)\deriv y=0
\end{equation*}
为全微分方程,则称$\mu(x,y)$为方程(\ref{total_equation})的积分因子.

\textbf{4.某些已知的二元函数的全微分公式}
\begin{align*}
    & x\deriv y+y\deriv x=\deriv(xy) && \dfrac{x\deriv y-y\deriv x}{x^2}=\deriv\left(\dfrac{y}{x}\right) \\
    & \dfrac{-x\deriv y+y\deriv x}{y^2}=\deriv\left(\dfrac{x}{y}\right) && \dfrac{x\deriv x+y\deriv y}{\sqrt{x^2+y^2}}=\deriv(\sqrt{x^2+y^2}) \\
    & \dfrac{-x\deriv y+y\deriv x}{xy}=\deriv\left(\ln\dfrac{x}{y}\right) && \dfrac{x\deriv y-y\deriv x}{x^2+y^2}=\deriv\left(\arctan\dfrac{y}{x}\right) \\
    & \dfrac{y\deriv x-x\deriv y}{x^2+y^2}=\deriv\left(\arctan\dfrac{x}{y}\right) && \dfrac{x\deriv x+y\deriv y}{x^2+y^2}=\dfrac{1}{2}\deriv\ln(x^2+y^2) \\
    & \dfrac{x\deriv y-y\deriv x}{y^2-x^2}=\deriv\ln\sqrt{\dfrac{y-x}{y+x}}
\end{align*}

\section{可降阶的高阶微分方程}
\textbf{1.$y^{(n)}=f(x)$}

方程特点是右端为自变量$x$的函数,且不含有函数$y$及其导数$y',y'',\cdots,y^{(n-1)}$,将方程两边对$x$逐次积分即得其通解
\begin{equation*}
    y=\int\deriv x\cdots\int f(x)\deriv x+\dfrac{C_1}{(n-1)!}x^{n-1}+\dfrac{C_2}{(n-2)!}x^{n-2}+\cdots+C_{n-1}x+C_n
\end{equation*}

\textbf{2.$y''=f(x,y')$}

方程特点是右端不显含函数$y$,令$y'=p,y''=\dfrac{\deriv p}{\deriv x}=p'$,代入原方程即可化为一阶方程$p'=f(x,p)$,若其解为$p=\varphi(x,C_1)$,则原方程的通解为
\begin{equation*}
    y=\int\varphi(x,C_1)\deriv x+C_2
\end{equation*}

\textbf{3.$y''=f(y,y')$}

方程特点是右端不显含自变量$x$,令$y'=p$,并利用复合函数的求导法则,把$y''$化为对$y$的导数,即
\begin{equation*}
    y''=\dfrac{\deriv p}{\deriv x}=\dfrac{\deriv p}{\deriv y}\cdot\dfrac{\deriv y}{\deriv x}=p\cdot\dfrac{\deriv p}{\deriv y}
\end{equation*}
代入原方程即可化为一阶方程
\begin{equation*}
    p\cdot\dfrac{\deriv p}{\deriv y}=f(y,p)
\end{equation*}
其解为$p=\varphi(y,C_1)$,即$\dfrac{\deriv y}{\deriv x}=\varphi(y,C_1)$,则原方程的通解为
\begin{equation*}
    \int\dfrac{\deriv y}{\varphi(y,C_1)}=x+C_2
\end{equation*}

\section{高阶线性微分方程解的结构}
$n$阶线性微分方程的一般形式为
\begin{equation}
    y^{(n)}+P_1(x)y^{(n-1)}+P_2(x)y^{(n-2)}+\cdots+P_{n-1}(x)y'+P_n(x)y=f(x) \label{n_th_order_linear_diff_equation}
\end{equation}
其中$n\geq2$,$P_1(x),P_2(x),\cdots,P_n(x),f(x)$为已知连续函数.

当$f(x)\equiv0$时,方程(\ref{n_th_order_linear_diff_equation})称为$n$阶线性齐次微分方程;当$f(x)\neq0$时,方程(\ref{n_th_order_linear_diff_equation})称为$n$阶线性非齐次微分方程.

现以二阶线性微分方程
\begin{align}
    & y''+P(x)y'+Q(x)y=0 \label{second_linear_diff_equation} \\
    & y''+P(x)y'+Q(x)y=f(x) \label{second_linear_diff_equation_nz}
\end{align}
为例,讨论其解的性质及其解法。这些性质及解法均可推广到任意高阶的线性微分方程.

(1)若函数$y_1(x),y_2(x)$是线性齐次方程(\ref{second_linear_diff_equation})的两个解,则$y=C_1y_1+C_2y_2$也是方程(\ref{second_linear_diff_equation})的解,其中$C_1,C_2$为任意常数.

(2)若$y_1(x),y_2(x)$是方程(\ref{second_linear_diff_equation})的两个线性无关的解,则$C_1y_1(x)+C_2y_2(x)$是方程(\ref{second_linear_diff_equation})的通解,其中$C_1,C_2$为任意常数.

(3)设$y^*$是线性非齐次方程(\ref{second_linear_diff_equation_nz})的一个特解,$Y=C_1y_1(x)+C_2y_2(x)$是对应的齐次方程(\ref{second_linear_diff_equation})的通解,则$y=Y+y^*$是非齐次方程(\ref{second_linear_diff_equation_nz})的通解.

(4)设线性非齐次方程(\ref{second_linear_diff_equation_nz})的右端$f(x)$是两个函数之和,如
\begin{equation*}
    y''+P(x)y'+Q(x)y=f_1(x)+f_2(x)
\end{equation*}
而$y_1(x)$与$y_2(x)$分别是方程$y''+P(x)y'+Q(x)y=f_1(x)$和$y''+P(x)y'+Q(x)y=f_2(x)$的解,则$y_1(x)+y_2(x)$是方程$y''+P(x)y'+Q(x)y=f_1(x)+f_2(x)$的解.

\section{常系数齐次线性微分方程}
\textbf{1.二阶常系数齐次线性微分方程的通解}

设
\begin{equation}
    y''+py'+qy=0 \label{second_order_constant_coefficient_linear_diff_equation}
\end{equation}

(1)特征方程有两个相异根$r_1\neq r_2$,则方程(\ref{second_order_constant_coefficient_linear_diff_equation})的通解为$Y=C_1e^{r_1x}+C_2e^{r_2x}$;

(2)特征方程有两个相等实根$r_1=r_2=r$,则方程(\ref{second_order_constant_coefficient_linear_diff_equation})的通解为$Y=(C_1+C_2x)e^{rx}$;

(3)特征方程有一对共轭复根$r_{1,2}=\alpha\pm i\beta$,则方程(\ref{second_order_constant_coefficient_linear_diff_equation})的通解为$Y=e^{\alpha x}(C_1\cos\beta x+C_2\sin\beta x)$.

\textbf{2.$n$阶常系数线性齐次微分方程}

设$n$阶常系数线性齐次微分方程是
\begin{equation}
    y^{(n)}+p_1(x)y^{(n-1)}+p_2y^{(n-2)}\cdots+p_{n-1}y'+p_ny=0
    \label{nth_order_constant_coefficient_linear_diff_equation}
\end{equation}
其中$p_1,p_2,\cdots,p_n$是常数,代数方程
\begin{equation*}
    r^n+p_1r^{n-1}+p_2r^{n-2}+\cdots+p_{n-1}r+p_n=0
\end{equation*}
成为微分方程(\ref{nth_order_constant_coefficient_linear_diff_equation})的特征方程,特征方程的根叫作微分方程(\ref{nth_order_constant_coefficient_linear_diff_equation})的特征根.

(1)如果$r_1$是特征方程的单根,则
\begin{equation*}
    y=Ce^{r_1x}
\end{equation*}
是微分方程(\ref{nth_order_constant_coefficient_linear_diff_equation})的解;

(2)如果特征方程有一对共轭复根$r_1=\alpha+i\beta,r_2=\alpha-i\beta$,则
\begin{equation*}
    y=e^{\alpha x}(C_1\cos\beta x+C_2\sin\beta x)
\end{equation*}
是微分方程(\ref{nth_order_constant_coefficient_linear_diff_equation})的解;

(3)如果$r_1$是特征方程的$k$重根,则
\begin{equation*}
    y=(C_1+C_2x+\cdots+C_kx^{k-1})e^{r_1x}
\end{equation*}
是微分方程(\ref{nth_order_constant_coefficient_linear_diff_equation})的解.

\section{常系数非齐次线性微分方程}
\textbf{1.二阶常系数非齐次线性微分方程的解法}

(4)如果$r_1=\alpha+i\beta,r_2=\alpha-i\beta$都是特征方程的$k$重根,则
\begin{equation*}
    y=e^{\alpha x}[(C_1+C_2x+\cdots+C_kx^{k-1})\cos\beta x+(D_1+D_2x+\cdots+D_kx^{k-1})\sin\beta x]
\end{equation*}
是微分方程(\ref{nth_order_constant_coefficient_linear_diff_equation})的解.

\section{常系数非齐次线性微分方程}
设二阶常系数非齐次线性微分方程为
\begin{equation}
    y''+py'+qy=f(x) \label{nonhomogeneous_second_order_constant_coefficient_linear_diff_equation}
\end{equation}
其中$p,q$为常数.

(1)如果方程(\ref{nonhomogeneous_second_order_constant_coefficient_linear_diff_equation})的右端$f(x)$是$x$的$n$次多项式$P_n(x)$,即$f(x)=P_n(x)$时,而常数0是特征方程的$k$重根时,可设特解为
\begin{equation*}
    y*=x^kQ_n(x)
\end{equation*}
其中$Q_n(x)$也是$x$的$n$次多项式,但其系数是待定的常数,如果常数0不是特征根,则取$k=0$.

(2)如果方程(\ref{nonhomogeneous_second_order_constant_coefficient_linear_diff_equation})的右端$f(x)=e^{\alpha x}P_n(x)$,可设特解为
\begin{equation*}
    y*=x^kQ_n(x)e^{\alpha x}
\end{equation*}
其中$Q_n(x)$(待定)是与$P_n(x)$同次的多项式,
\begin{equation*}
    k=
    \begin{cases}
        0, \quad \alpha\mbox{不是特征方程的根} \\
        1, \quad \alpha\mbox{是特征方程的单根} \\
        2, \quad \alpha\mbox{是特征方程的二重根}
    \end{cases}
\end{equation*}

(3)如果方程(\ref{nonhomogeneous_second_order_constant_coefficient_linear_diff_equation})的右端$f(x)=e^{\alpha x}[P_l(x)\cos\beta x+P_n(x)\sin\beta x]$,其中$P_l(x),P_n(x)$分别是$x$的$l$次和$n$次多项式,$\alpha,\beta$是已知常数,设特解形式为
\begin{equation*}
    y*=x^ke^{\alpha x}[R_m^{(1)}(x)\cos\beta x+R_m^{(2)}(x)\sin\beta x]
\end{equation*}
其中$R_m^{(1)}(x),R_m^{(2)}(x)$是两个$m$次多项式,
\begin{align*}
    & m=\max\{l,n\} \\
    & k=
    \begin{cases}
        0, \quad \alpha+i\beta\mbox{不是特征方程的根} \\
        1, \quad \alpha+i\beta\mbox{是特征方程的单复根}
    \end{cases}
\end{align*}