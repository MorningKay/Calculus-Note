\setcounter{chapter}{9}

\chapter{曲线积分与曲面积分}

\begin{introduction}
    \item 对弧长的曲线积分
    \item 对坐标的曲线积分
    \item 格林公式
    \item 对面积的曲面积分
    \item 对坐标的曲面积分
    \item 高斯公式、通量和散度
    \item 斯托克斯公式、环流量和旋度
\end{introduction}

\section{对弧长的曲线积分}

\textbf{对弧长的曲线积分的概念}(又称第一类曲线积分)

\begin{equation*}
    \int_{L}f(x,y)\deriv s=\lim_{\lambda\rightarrow 0}\sum_{i=1}^{n}f(\xi_i, \eta_i)\Delta s_{i}
\end{equation*}

如果函数$f(x,y)$在曲线$L$上连续,则$f(x,y)$在曲线$L$上对弧长的曲线积分$\int_{L}f(x,y)\deriv s$一定存在.

上述概念可以推广到空间,如果$f(x,y,z)$是定义在空间中分段光滑曲线$L$上的有界函数,则函数$f(x,y,z)$在曲线$L$上对弧长的曲线积分是
\begin{equation*}
    \int_{L}f(x,y,z)\deriv s=\lim_{\lambda\rightarrow 0}\sum_{i=1}^{n}f(\xi_i, \eta_i, \zeta_i)\Delta s_{i}
\end{equation*}

\begin{property}\label{property: line_integral}
    \begin{enumerate}
        \item $\int_{L}[f_1(x,y)\pm f_2(x,y)]\deriv s=\int_{L}f_1(x,y)\deriv s\pm\int_{L}f_2(x,y)\deriv s$
        \item $\int_{L}k\cdot f(x,y)\deriv s=k\int_{L}f(x,y)\deriv s$,其中$k$为常数
        \item 若$L=L_1+L_2$,则
        \begin{equation*}
            \int_{L}f(x,y)\deriv s=\int_{L_1}f(x,y)\deriv s+\int_{L_2}f(x,y)\deriv s
        \end{equation*}
    \end{enumerate}
\end{property}

\textbf{对弧长曲线积分的计算法}

(1)设函数$f(x,y)$在平面曲线
\begin{equation*}
    L:
    \begin{cases}
        x=x(t) \\
        y=y(t)
    \end{cases}
    \quad (\alpha\leq t\leq \beta)
\end{equation*}
上连续,$x'(t),y'(t)$在区间$[\alpha,\beta]$上连续,则
\begin{equation*}
    \int_{L}f(x,y)\deriv s=\int_{\alpha}^{\beta}f[x(t),y(t)]\sqrt{[x'(t)]^{2}+[y'(t)]^{2}}\deriv t
\end{equation*}

如果曲线$L$的方程为$y=y(x),(a\leq x\leq b)$且$y'(x)$在区间$[a,b]$上连续,则
\begin{equation*}
    \int_{L}f(x,y)\deriv s=\int_{a}^{b}f[x,y(x)]\sqrt{1+[y'(x)]^{2}}\deriv x
\end{equation*}

(2)设函数$f(x,y,z)$在空间曲线
\begin{equation*}
    L:
    \begin{cases}
        x=x(t) \\
        y=y(t) \\
        z=z(t)
    \end{cases}
    \quad (\alpha\leq t\leq \beta)
\end{equation*}
上连续,$x'(t),y'(t),z'(t)$在区间$[\alpha,\beta]$上连续,则
\begin{equation*}
    \int_{L}f(x,y,z)\deriv s=\int_{\alpha}^{\beta}f[x(t),y(t),z(t)]\sqrt{[x'(t)]^{2}+[y'(t)]^{2}+[z'(t)]^{2}}\deriv t
\end{equation*}

\textbf{普通对称性与轮换对称性}

\textbf{(1)普通对称性}

假设$\Gamma$关于$yOz$面对称,则
\begin{equation*}
    \int_{\Gamma}f(x,y,z)\deriv s=
    \begin{cases}
        2\displaystyle\int_{\Gamma_1}f(x,y,z)\deriv s, \quad & f(x,y,z)=f(-x,y,z) \\
        0, \quad & f(x,y,z)=-f(-x,y,z)
    \end{cases}
\end{equation*}
其中$\Gamma_1$是$\Gamma$在$yOz$面前面的部分.

关于其他坐标面对称的情况与此类似.

\textbf{(2)轮换对称性}

若把$x$与$y$对调后,$\Gamma$不变,则$\displaystyle\int_{\Gamma}f(x,y,z)\deriv s=\int_{\Gamma}f(y,x,z)\deriv s$,这就是轮换对称性.

关于其他情况与此类似.

\section{对坐标的曲线积分}

\textbf{对坐标的曲线积分的概念}(又称第二类曲线积分)

\begin{align*}
    & \int_{L}P(x,y)\deriv x=\lim_{\lambda\rightarrow 0}\sum_{i=1}^{n} P(\xi_i, \eta_i)\Delta x_{i} \\
    & \int_{L}Q(x,y)\deriv y=\lim_{\lambda\rightarrow 0}\sum_{i=1}^{n} Q(\xi_i, \eta_i)\Delta y_{i}
\end{align*}

如果函数$P(x,y)$和$Q(x,y)$在有向曲线$L$上连续,上述积分都存在.

类似地,在空间有向曲线$\Gamma$上对坐标$x,y,z$的曲线积分
\begin{align*}
    & \int_{\Gamma}P(x,y,z)\deriv x=\lim_{\lambda\rightarrow 0}\sum_{i=1}^{n} P(\xi_i, \eta_i, \zeta_i)\Delta x_{i} \\
    & \int_{\Gamma}Q(x,y,z)\deriv y=\lim_{\lambda\rightarrow 0}\sum_{i=1}^{n} Q(\xi_i, \eta_i, \zeta_i)\Delta y_{i} \\
    & \int_{\Gamma}R(x,y,z)\deriv z=\lim_{\lambda\rightarrow 0}\sum_{i=1}^{n} R(\xi_i, \eta_i, \zeta_i)\Delta z_{i}
\end{align*}

\begin{property}
    $\int_{\overset{\Large\frown}{AB}}P\deriv x+Q\deriv y=-\int_{\overset{\Large\frown}{BA}}P\deriv x+Q\deriv y$
\end{property}

\textbf{对坐标的曲线积分的计算法}

(1)设函数$P(x,y)$和$Q(x,y)$在有向曲线$L$上连续,$L$的参数方程为
\begin{equation*}
    L:
    \begin{cases}
        x=x(t) \\
        y=y(t)
    \end{cases}
    \quad (\alpha\leq t\leq \beta)
\end{equation*}
且$x'(t),y'(t)$连续,而$t=\alpha$时对应于起点$A$,$t=\beta$对应于终点$B$,则
\begin{align*}
    & \int_{\overset{\Large\frown}{AB}}P(x,y)\deriv x=\int_{\alpha}^{\beta} P[x(t),y(t)]x'(t)\deriv t \\
    & \int_{\overset{\Large\frown}{AB}}Q(x,y)\deriv y=\int_{\alpha}^{\beta} Q[x(t),y(t)]y'(t)\deriv t
\end{align*}

(2)如果曲线$L$是由方程$y=y(x)(a\leq x\leq b)$给出,曲线$L$的起点$A$的横坐标为$x=a$,终点$B$的横坐标为$x=b$,函数$y(x)$具有连续的一阶导数,则
\begin{align*}
    & \int_{\overset{\Large\frown}{AB}}P(x,y)\deriv x=\int_{a}^{b} P[x,y(x)]\deriv x \\
    & \int_{\overset{\Large\frown}{AB}}Q(x,y)\deriv y=\int_{a}^{b} Q[x,y(x)]y'(x)\deriv x
\end{align*}

(3)如果曲线$L$是由方程$x=x(y)(c\leq y\leq d)$给出,曲线$L$的起点$A$的纵坐标为$y=c$,终点$B$的纵坐标为$y=d$,函数$x(y)$具有连续的一阶导数,则
\begin{align*}
    & \int_{\overset{\Large\frown}{AB}}P(x,y)\deriv x=\int_{c}^{d} P[x(y),y]x'(y)\deriv y \\
    & \int_{\overset{\Large\frown}{AB}}Q(x,y)\deriv y=\int_{c}^{d} Q[x(y),y]\deriv y
\end{align*}

(4)对于空间曲线积分,如果函数$P(x,y,z)$,$Q(x,y,z)$,$R(x,y,z)$在有向曲线$\Gamma$上连续,$\Gamma$的参数方程为
\begin{equation*}
    \Gamma:
    \begin{cases}
        x=x(t) \\
        y=y(t) \\
        z=z(t)
    \end{cases}
    \quad (\alpha\leq t\leq \beta)
\end{equation*}
而$x'(t),y'(t),z'(t)$连续,且$t=\alpha$时对应于起点$A$,$t=\beta$对应于终点$B$,则
\begin{align*}
    & \int_{\Gamma}P(x,y,z)\deriv x=\int_{\alpha}^{\beta} P[x(t),y(t),z(t)]x'(t)\deriv t \\
    & \int_{\Gamma}Q(x,y,z)\deriv y=\int_{\alpha}^{\beta} Q[x(t),y(t),z(t)]y'(t)\deriv t \\
    & \int_{\Gamma}R(x,y,z)\deriv z=\int_{\alpha}^{\beta} R[x(t),y(t),z(t)]z'(t)\deriv t
\end{align*}

\textbf{两类曲线积分的关系}

(1)设平面上有向曲线$L$上任一点$M(x,y)$处与$L$方向一致的切线的方向余弦为
\begin{equation*}
    \cos\alpha=\frac{\deriv x}{\deriv s},\quad \cos\beta=\frac{\deriv y}{\deriv s}
\end{equation*}
则
\begin{equation*}
    \int_{L}P\deriv x+Q\deriv y=\int_{L}(P\cos\alpha+Q\cos\beta)\deriv s
\end{equation*}

(2)设空间有向曲线$\Gamma$上任一点$N(x,y,z)$处与$\Gamma$方向一致的切线的方向余弦为
\begin{equation*}
    \cos\alpha=\frac{\deriv x}{\deriv s},\quad \cos\beta=\frac{\deriv y}{\deriv s},\quad \cos\gamma=\frac{\deriv z}{\deriv s}
\end{equation*}
则
\begin{equation*}
    \int_{\Gamma}P\deriv x+Q\deriv y+R\deriv z=\int_{\Gamma}(P\cos\alpha+Q\cos\beta+R\cos\gamma)\deriv s
\end{equation*}

\section{格林公式及其应用}

\textbf{格林公式}

设函数$P(x,y)$,$Q(x,y)$在平面区域$D$及其边界曲线$L$上具有连续的一阶偏导数,则
\begin{equation*}
    \oint_{L}P\deriv x+Q\deriv y=\iint \limits_{D}\left(\frac{\partial Q}{\partial x}-\frac{\partial P}{\partial y}\right)\deriv x\deriv y
\end{equation*}
其中$L$取正向.

\textbf{平面上曲线积分与路径无关的条件}

设函数$P(x,y)$,$Q(x,y)$在平面单连通区域$D$内具有连续的一阶偏导数,则下面四个命题等价
\begin{enumerate}
    \item 曲线$L(\overset{\Large\frown}{AB})$是$D$内由点$A$到点$B$的一段有向曲线,则曲线积分
    \begin{equation*}
        \int_{L}P\deriv x+Q\deriv y
    \end{equation*}
    与路径无关,只与起点$A$和终点$B$有关.

    \item 在区域$D$内沿任意一条闭曲线$L$的曲线积分有
    \begin{equation*}
        \oint_{L}P\deriv x+Q\deriv y=0
    \end{equation*}

    \item 在区域$D$内任意一点$(x,y)$处有
    \begin{equation*}
        \dfrac{\partial Q}{\partial x}=\dfrac{\partial P}{\partial y}
    \end{equation*}

    \item 在区域$D$内存在函数$u(x,y)$,使得$P\deriv x+Q\deriv y$是该二元函数$u(x,y)$的全微分,且有
    \begin{equation*}
        u(x,y)=\int_{(x_0,y_0)}^{(x,y)}P\deriv x+Q\deriv y
    \end{equation*}
    其中$(x_0,y_0)$是区域$D$内的某一定点,$(x,y)$是$D$内的任意一点.
\end{enumerate}

\section{对面积的曲面积分}

\textbf{对面积的曲面积分的概念}(又称第一类曲面积分)
\begin{equation*}
    \iint \limits_{\Sigma}f(x,y,z)\deriv S=\lim_{\lambda\rightarrow 0}\sum_{i=1}^{n}f(\xi_i,\eta_i,\zeta_i)\Delta S_i
\end{equation*}

\textbf{对面积的曲面积分计算法}

设光滑曲面$\Sigma$的方程为$z=z(x,y)$,$\Sigma$在$xOy$平面上的投影域为$D_{xy}$,函数$z=z(x,y)$具有一节连续的偏导数,被积函数$f(x,y,z)$在$\Sigma$上连续,则
\begin{equation*}
    \iint \limits_{\Sigma}f(x,y,z)\deriv S=\iint \limits_{D_{xy}}f[x,y,z(x,y)]\sqrt{1+\left(\frac{\partial z}{\partial x}\right)^{2}+\left(\frac{\partial z}{\partial y}\right)^{2}}\deriv x\deriv y
\end{equation*}

当光滑曲面$\Sigma$的方程为$x=x(y,z)$或$y=y(x,z)$时,可以把曲面积分化为相应的二重积分
\begin{equation*}
    \iint \limits_{\Sigma}f(x,y,z)\deriv S=\iint \limits_{D_{yz}}f[x(y,z),y,z]\sqrt{1+\left(\frac{\partial x}{\partial y}\right)^{2}+\left(\frac{\partial x}{\partial z}\right)^{2}}\deriv y\deriv z
\end{equation*}
或
\begin{equation*}
    \iint \limits_{\Sigma}f(x,y,z)\deriv S=\iint \limits_{D_{xz}}f[x,y(x,z),z]\sqrt{1+\left(\frac{\partial y}{\partial x}\right)^{2}+\left(\frac{\partial y}{\partial z}\right)^{2}}\deriv x\deriv z
\end{equation*}
其中$D_{yz}$,$D_{xz}$分别是$\Sigma$在$yz$平面和$xz$平面上的投影域.

\textbf{普通对称性与轮换对称性}

\textbf{(1)普通对称性}

假设$\Sigma$关于$yOz$面对称,则
\begin{equation*}
    \iint\limits_{\Sigma}f(x,y,z)\deriv S=
    \begin{cases}
        2\displaystyle\iint\limits_{\Sigma_1}f(x,y,z)\deriv S, \quad & f(x,y,z)=f(-x,y,z) \\
        0, \quad & f(x,y,z)=-f(-x,y,z)
    \end{cases}
\end{equation*}
其中$\Sigma_1$是$\Sigma$在$yOz$面前面的部分.

关于其他坐标面对称的情况与此类似.

\textbf{(2)轮换对称性}

若把$x$与$y$对调后,$\Sigma$不变,则$\displaystyle\iint\limits_{\Sigma}f(x,y,z)\deriv S=\iint\limits_{\Sigma}f(y,x,z)\deriv S$,这就是轮换对称性.

关于其他情况与此类似.

\section{对坐标的曲面积分}

\textbf{对坐标的曲面积分的概念}(又称第二类曲面积分)

有向曲面 \quad 通常遇到的曲面都是双侧的,规定了正侧的曲面称为有向曲面

设$\Sigma$为光滑的有向曲面,$P(x,y,z)$,$Q(x,y,z)$,$R(x,y,z)$都是定义在$\Sigma$上的有界函数,将曲面$\Sigma$任意分成$n$个小曲面$\Delta S_i(i=1,2,\cdots,n)$,在每个小曲面上任取一点$N_i(\xi_i,\eta_i,\zeta_i)$,曲面$\Sigma$的正侧在点$N_i$处的法向量为
\begin{equation*}
    \bm{n}_i=\cos\alpha_i\bm{i}+\cos\beta_i\bm{j}+\cos\gamma_i\bm{k}
\end{equation*}
有向小曲面$\Delta S_i$在$xOy$平面上投影为$\Delta S_{i,xy}=\Delta S_i\cos\gamma_i$,如果当各小曲面直径的最大值$\lambda\rightarrow 0$时,和式$\sum_{i=1}^n R(\xi_i,\eta_i,\zeta_i)\Delta S_{i,xy}$的极限存在,则称此极限为函数$R(x,y,z)$在有向曲面$\Sigma$的正侧上对坐标$x,y$的曲面积分,记为$\iint \limits_{\Sigma}R(x,y,z)\deriv x\deriv y$,即
\begin{equation*}
    \iint \limits_{\Sigma}R(x,y,z)\deriv x\deriv y=\lim_{\lambda\rightarrow 0}\sum_{i=1}^n R(\xi_i,\eta_i,\zeta_i)\Delta S_{i,xy}
\end{equation*}

类似地,函数$P(x,y,z)$在有向曲面$\Sigma$的正侧上对坐标$y,z$的曲面积分
\begin{equation*}
    \iint \limits_{\Sigma}P(x,y,z)\deriv y\deriv z=\lim_{\lambda\rightarrow 0}\sum_{i=1}^n P(\xi_i,\eta_i,\zeta_i)\Delta S_{i,yz}
\end{equation*}
其中$\Delta S_{i,yz}=\Delta S_i\cos\alpha_i$.

函数$Q(x,y,z)$在有向曲面$\Sigma$的正侧上对坐标$x,z$的曲面积分
\begin{equation*}
    \iint \limits_{\Sigma}Q(x,y,z)\deriv x\deriv z=\lim_{\lambda\rightarrow 0}\sum_{i=1}^n Q(\xi_i,\eta_i,\zeta_i)\Delta S_{i,xz}
\end{equation*}
其中$\Delta S_{i,xz}=\Delta S_i\cos\beta_i$.

\begin{property}
    若$\Sigma$表示有向曲面的正侧,该曲面的另一侧为负侧记为$\Sigma^{-}$,则有
    \begin{equation*}
        \iint \limits_{\Sigma}P\deriv y\deriv z+Q\deriv x\deriv z+R\deriv x\deriv y=-\iint \limits_{\Sigma^{-}}P\deriv y\deriv z+Q\deriv x\deriv z+R\deriv x\deriv y
    \end{equation*}
    即当积分曲面改变为相反侧时,对坐标的曲面积分要改变符号
\end{property}

\textbf{对坐标的曲面积分的计算法}

设光滑曲面$\Sigma$是由方程$z=z(x,y)$所给出的曲面上侧,角$\gamma$是曲面$\Sigma$的法向量$\bm{n}$与$z$轴的夹角,此时$\cos\gamma>0$,曲面$\Sigma$在$xOy$平面上的投影区域为$D_{xy}$,函数$z=z(x,y)$在$D_{xy}$上具有一阶连续偏导数,被积函数$R(x,y,z)$在$\Sigma$上连续,则
\begin{equation*}
    \iint \limits_{\Sigma}R(x,y,z)\deriv x\deriv y=\iint \limits_{D_{xy}}R[x,y,z(x,y)]\deriv x\deriv y
\end{equation*}
如果积分曲面取在$\Sigma$的下侧,此时$\cos\gamma<0$,则
\begin{equation*}
    \iint \limits_{\Sigma}R(x,y,z)\deriv x\deriv y=-\iint \limits_{D_{xy}}R[x,y,z(x,y)]\deriv x\deriv y
\end{equation*}

当曲面$\Sigma$是母线平行于$z$轴的柱面$F(x,y)=0$时,此时$\cos\gamma=\cos\dfrac{\pi}{2}=0$,则
\begin{equation*}
    \iint \limits_{\Sigma}R(x,y,z)\deriv x\deriv y=0
\end{equation*}
类似地有
\begin{align*}
    & \iint \limits_{\Sigma}P(x,y,z)\deriv y\deriv z=\pm\iint \limits_{D_{yz}}P[x(y,z),y,z]\deriv y\deriv z\\
    & \iint \limits_{\Sigma}Q(x,y,z)\deriv x\deriv z=\pm\iint \limits_{D_{xz}}Q[x,y(x,z),z]\deriv x\deriv z
\end{align*}

\textbf{两类曲面积分之间的关系}

设曲面$\Sigma$上任一点$(x,y,z)$处法向量$\bm{n}$的方向余弦为$(\cos\alpha,\cos\beta,\cos\gamma)$,则有
\begin{equation*}
    \iint \limits_{\Sigma}P\deriv y\deriv z+Q\deriv x\deriv z+R\deriv x\deriv y=\iint \limits_{\Sigma}\left(P\cos\alpha+Q\cos\beta+R\cos\gamma\right)\deriv S
\end{equation*}

\section{高斯公式 \quad 通量与散度}
\textbf{1.高斯(Gauss)公式}

设空间闭区域$\Omega$是由分片光滑的闭区域$\Sigma$所围成,函数$P(x,y,z)$、$Q(x,y,z)$、$R(x,y,z)$在$\Omega$及其边界曲面$\Sigma$上具有连续的一阶偏导数,则
\begin{equation*}
    \oiint \limits_{\Sigma}P\deriv y\deriv z+Q\deriv x\deriv z+R\deriv x\deriv y=\iiint \limits_{\Omega}\left(\frac{\partial P}{\partial x}+\frac{\partial Q}{\partial y}+\frac{\partial R}{\partial z}\right)\deriv x\deriv y\deriv z
\end{equation*}
或
\begin{equation*}
    \oiint \limits_{\Sigma}(P\cos\alpha+Q\cos\beta+R\cos\gamma)\deriv S=\iiint \limits_{\Omega}\left(\frac{\partial P}{\partial x}+\frac{\partial Q}{\partial y}+\frac{\partial R}{\partial z}\right)\deriv x\deriv y\deriv z
\end{equation*}
其中$\Sigma$取外侧,$\cos\alpha,\cos\beta,\cos\gamma$为$\Sigma$上任一点$(x,y,z)$处外法线向量的方向余弦

\textbf{2.通量与散度}

设向量场
\begin{equation*}
    \bm{A}(x,y,z)=P(x,y,z)\bm{i}+Q(x,y,z)\bm{j}+R(x,y,z)\bm{k}
\end{equation*}
其中$P,Q,R$具有连续的一阶偏导数,$\Sigma$是场内的一个有向曲面,则称
\begin{equation*}
    \Phi=\iint \limits_{\Sigma}\bm{A}\cdot\deriv \bm{S}=\iint \limits_{\Sigma}P\deriv y\deriv z+Q\deriv x\deriv z+R\deriv x\deriv y
\end{equation*}
为向量场$\bm{A}$通过曲面$\Sigma$的通量(或流量)

$\dfrac{\partial P}{\partial x}+\dfrac{\partial Q}{\partial y}+\dfrac{\partial R}{\partial z}$称为向量场$\bm{A}$的散度,记为$\text{div}\bm{A}$,即
\begin{equation*}
    \text{div}\bm{A}=\dfrac{\partial P}{\partial x}+\dfrac{\partial Q}{\partial y}+\dfrac{\partial R}{\partial z}
\end{equation*}

有了散度的概念,高斯公式可写成
\begin{equation*}
    \oiint \limits_{\Sigma}\bm{A}\cdot\deriv \bm{S}=\iiint \limits_{\Omega}\text{div}\bm{A}\deriv V
\end{equation*}
其中$\Sigma$是空间闭区域$\Omega$的边界曲面的外侧.

\section{斯托克斯公式 \quad 环流量与旋度}
\textbf{1.斯托克斯(Stokes)公式}

设函数$P(x,y,z)$、$Q(x,y,z)$、$R(x,y,z)$在包含曲面$S$的空间域$\Omega$内具有连续的一阶偏导数,$L$是曲面$\Sigma$的边界曲线,则
\begin{equation*}
    \oint_{L}P\deriv x+Q\deriv y+R\deriv z=\iint \limits_{\Sigma}
    \begin{vmatrix}
        \deriv y\deriv z & \deriv x\deriv z & \deriv x\deriv y \\
        \dfrac{\partial}{\partial x} & \dfrac{\partial}{\partial y} & \dfrac{\partial}{\partial z} \\
        P & Q & R
    \end{vmatrix}
    \deriv S
    = \iint \limits_{\Sigma}
    \begin{vmatrix}
        \cos\alpha & \cos\beta & \cos\gamma \\
        \dfrac{\partial}{\partial x} & \dfrac{\partial}{\partial y} & \dfrac{\partial}{\partial z} \\
        P & Q & R
    \end{vmatrix}
    \deriv S
\end{equation*}
其中$L$的正向与$\Sigma$所取的正侧符合右手法则,$\cos\alpha,\cos\beta,\cos\gamma$为曲面$S$的正侧上任一点$(x,y,z)$处法向量$\bm{n}$的方向余弦

\textbf{2.环流量与旋度}

设向量场
\begin{equation*}
    \bm{A}(x,y,z)=P(x,y,z)\bm{i}+Q(x,y,z)\bm{j}+R(x,y,z)\bm{k}
\end{equation*}
$L$是场内的一条有向闭曲线,则称
\begin{equation*}
    \Gamma=\oint_{L}\bm{A}\cdot\deriv \bm{r}=\oint_{L}P\deriv x+Q\deriv y+R\deriv z
\end{equation*}
为向量场$\bm{A}$沿曲线$L$的环流量,并称向量
\begin{equation*}
    \left(\frac{\partial R}{\partial y}-\frac{\partial Q}{\partial z}\right)\bm{i}+\left(\frac{\partial P}{\partial z}-\frac{\partial R}{\partial x}\right)\bm{j}+\left(\frac{\partial Q}{\partial x}-\frac{\partial P}{\partial y}\right)\bm{k}
\end{equation*}
为向量$\bm{A}$的旋度,记作$\text{rot}\bm{A}$,即
\begin{equation*}
    \text{rot}\bm{A}=\left(\frac{\partial R}{\partial y}-\frac{\partial Q}{\partial z}\right)\bm{i}+\left(\frac{\partial P}{\partial z}-\frac{\partial R}{\partial x}\right)\bm{j}+\left(\frac{\partial Q}{\partial x}-\frac{\partial P}{\partial y}\right)\bm{k}=
    \begin{vmatrix}
        \bm{i} & \bm{j} & \bm{k} \\
        \dfrac{\partial}{\partial x} & \dfrac{\partial}{\partial y} & \dfrac{\partial}{\partial z} \\
        P & Q & R
    \end{vmatrix}
\end{equation*}

有了旋度的概念,斯托克斯公式可写成
\begin{equation*}
    \oint_{L}\bm{A}\cdot\deriv \bm{r}=\iint \limits_{\Sigma}\text{rot}\bm{A}\cdot\deriv \bm{S}
\end{equation*}
其中
\begin{align*}
    & \deriv \bm{r}=\bm{i}\deriv x+\bm{j}\deriv y+\bm{k}\deriv z \\
    & \deriv \bm{S}=\bm{i}\deriv x\deriv y+\bm{j}\deriv x\deriv z+\bm{k}\deriv y\deriv z
\end{align*}
