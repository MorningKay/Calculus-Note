\setcounter{chapter}{2}

\chapter{微分中值定理和导数的应用}

\begin{introduction}
    \item 微分中值定理
    \item 洛必达法则
    \item 泰勒公式
    \item 单调性、凹凸性、极值、最值
    \item 曲率\ref{def:curvature}
\end{introduction}

\section{微分中值定理}
\begin{theorem}[罗尔中值定理] \label{thm:Rolle's_theorem}
    假设$f(x)$满足以下三个条件:
    \begin{enumerate}
        \item $f$在闭区间$[a.b]$上连续
        \item $f$在开区间$(a,b)$上可导
        \item $f(a)=f(b)$
    \end{enumerate}

    则至少存在一点$c\in(a,b)$使得$f'(c)=0$
\end{theorem}

\begin{theorem}[拉格朗日中值定理] \label{thm:the_mean_value_theorem}
    假设$f(x)$满足以下三个条件:
    \begin{enumerate}
        \item $f$在闭区间$[a.b]$上连续
        \item $f$在开区间$(a,b)$上可导
    \end{enumerate}

    则至少存在一点$c\in(a,b)$使得
    \begin{equation}
        f'(c) = \dfrac{f(b)-f(a)}{b-a}
        \nonumber
    \end{equation}

    或写作
    \begin{equation}
        f(b)-f(a) = f'(c)(b-a)
        \nonumber
    \end{equation}
\end{theorem}

\begin{theorem}[柯西中值定理] \label{thm:Cauchy_mean_value_theorem}
    假设$f(x),g(x)$满足以下三个条件:
    \begin{enumerate}
        \item $f,g$在闭区间$[a.b]$上连续
        \item $f,g$在开区间$(a,b)$上可导
        \item $g'$在$(a,b)$内每一点处均不为0
    \end{enumerate}

    则至少存在一点$c\in(a,b)$使得
    \begin{equation}
        \dfrac{f(b)-f(a)}{g(b)-g(a)} = \dfrac{f'(c)}{g'(c)}
        \nonumber
    \end{equation}
\end{theorem}

\section{洛必达法则}
设函数$f(x)$与$g(x)$满足:
\begin{enumerate}
    \item 在点$x_0$的某一邻域内(点$x_0$可除外)有定义,且$\displaystyle \lim_{x\rightarrow x_0}f(x)=0, \lim_{x\rightarrow x_0}g(x)=0$
    
    (或$\displaystyle \lim_{x\rightarrow x_0}f(x)=\infty, \lim_{x\rightarrow x_0}g(x)=\infty$)

    \item 在该邻域内可导,且$g'(x)\neq 0$

    \item $\displaystyle \lim_{x\rightarrow x_0} \dfrac{f'(x)}{g'(x)}$存在(或为$\infty$)。则$\displaystyle \lim_{x\rightarrow x_0}\dfrac{f(x)}{g(x)}=\lim_{x\rightarrow x_0}\dfrac{f'(x)}{g'(x)}$(或为$\infty$)
\end{enumerate}

以上两法则对于$x\rightarrow\infty$时的未定式$\dfrac{0}{0},\dfrac{\infty}{\infty}$同样适用。

\section{泰勒公式}
\textbf{泰勒定理} \quad 若$f(x)$在含有$x_0$的某个邻域内具有直到$n+1$阶的导数,则对于该邻域内任意点$x$,有泰勒公式
\begin{equation}
    f(x)=\sum_{k=0}^n \dfrac{f^{(k)}(x_0)}{k!}(x-x_0)^k+ R_n(x)
    \nonumber
\end{equation}

其中
\begin{equation}
    R_n(x) = \dfrac{f^{(n+1)}(\xi)}{(n+1)!}(x-x_0)^{n+1}
    \nonumber
\end{equation}

$\xi$介于$x_0$与$x$之间,$f^{(0)}(x_0)=f(x_0)$。此公式称为$f(x)$在$x_0$处(或按$x-x_0$的幂展开)的带有拉格朗日余项的$n$阶泰勒公式,$R_n(x)$的表达式称为拉格朗日余项。

\begin{proof}
    记$R_n(x)=f(x)-p_n(x)$,只需证明
    \begin{equation}
        R_n(x) = \dfrac{f^{(n+1)}(\xi)}{(n+1)!}(x-x_0)^{n+1} \quad (\xi\mbox{在}x_0\mbox{与}x\mbox{之间})
        \nonumber
    \end{equation}

    由假设可知,$R_n(x)$在$U(x_0)$内具有$n+1$阶导数,且
    \begin{equation}
        R_n(x_0)=R_n'(x_0)=R_n''(x_0)=\cdots=R_n^{(n)}(x_0)=0
        \nonumber
    \end{equation}

    对两个函数$R_n(x)$及$(x-x_0)^{n+1}$在以$x_0$及$x$为端点的区间上应用柯西中值定理(显然,这两个函数满足柯西中值定理的条件),得
    \begin{equation}
        \dfrac{R_n(x)}{(x-x_0)^{n+1}}=\dfrac{R_n(x)-R_n(x_0)}{(x-x_0)^{n+1}-0}=
        \dfrac{R_n'(\xi_1)}{(n+1)(\xi_1-x_0)^n} \quad (\xi_1\mbox{在}x_0\mbox{与}x\mbox{之间})
        \nonumber
    \end{equation}

    再对两个函数$R_n'(x)$及$(n+1)(x-x_0)^n$在以$x_0$及$\xi_1$为端点的区间上应用柯西中值定理,得
    \begin{equation}
        \dfrac{R_n'(\xi_1)}{(n+1)(\xi_1-x_0)^n}=\dfrac{R_n'(\xi_1)-R_n'(x_0)}{(n+1)(\xi_1-x_0)^n -0}=
        \dfrac{R_n''(\xi_2)}{(n+1)n(\xi_2-x_0)^{n-1}} \quad (\xi_2\mbox{在}x_0\mbox{与}\xi_1\mbox{之间})
        \nonumber
    \end{equation}

    照此方法继续做下去,经过$n+1$次后,得
    \begin{equation}
        \dfrac{R_n(x)}{(x-x_0)^{n+1}}=\dfrac{R_n^{(n+1)}(\xi)}{(n+1)!} \quad 
        (\xi\mbox{在}x_0\mbox{与}\xi_n\mbox{之间},\mbox{因而也在}x_0\mbox{与}x\mbox{之间})
        \nonumber
    \end{equation}

    注意到$R_n^{(n+1)}(x)=f^{(n+1)}(x)$(因$p_n^{(n+1)}(x)=0$),则由上式得
    \begin{equation}
        R_n(x)=\dfrac{f^{(n+1)}(\xi)}{(n+1)!}(x-x_0)^{n+1} \quad (\xi\mbox{在}x_0\mbox{与}x\mbox{之间})
        \nonumber
    \end{equation}

    定理证毕。
\end{proof}

\vspace{2mm}
\textbf{麦克劳林公式} \quad 在$x=0$展开的泰勒公式,也称为麦克劳林公式,即
\begin{equation}
    f(x)=\sum_{k=0}^n \dfrac{f^{(k)}(0)}{k!}x^k+
    \dfrac{f^{(n+1)}(\xi)}{(n+1)!}x^{n+1}
    \nonumber
\end{equation}

其中$\xi$介于$x_0$与$x$之间。

\textbf{常用的泰勒展开式}
\begin{equation}
    \begin{aligned}
        &e^x = 1+x+\dfrac{x^2}{2!}+\cdots+\dfrac{x^n}{n!}+o(x^n)\\
        &\sin x = x-\dfrac{x^3}{3!}+\dfrac{x^5}{5!}-\cdots+(-1)^n\dfrac{x^{2n+1}}{(2n+1)!}+o(x^{2n+1})\\
        &\cos x = 1-\dfrac{x^2}{2!}+\dfrac{x^4}{4!}-\dfrac{x^6}{6!}+\cdots+(-1)^n\dfrac{x^{2n}}{(2n)!}+o(x^{2n})\\
        &\ln(1+x) = x-\dfrac{x^2}{2}+\dfrac{x^3}{3}-\cdot+(-1)^n\dfrac{x^{n+1}}{n+1}+o(x^{n+1})\\
        &\dfrac{1}{1-x} = 1+x+x^2+\cdots+x^n+o(x^n)\\
        &\dfrac{1}{1+x} = 1-x+x^2-\cdots+(-1)^n x^n+o(x^n)\\
        &(1+x)^m = 1+mx+\dfrac{m(m-1)}{2!}x^2+\cdots+\dfrac{m(m-1)\cdot \cdots \cdot (m-n+1)}{n!}x^n+o(x^n)
    \end{aligned}
    \nonumber
\end{equation}

\section{函数的单调性与曲线的凹凸性}
\textbf{1.函数的单调性}

设函数$y=f(x)$在$[a,b]$上连续,在$(a,b)$内可导

(1)若在$(a,b)$内$f'(x)>0$,则$f(x)$在$[a,b]$上单调增加;

(2)若在$(a,b)$内$f'(x)<0$,则$f(x)$在$[a,b]$上单调减少。

\textbf{求$y=f(x)$的单调区间步骤:}

(1)明确定义域并找出无定义端点;

(2)找出使$f'(x)=0$的点(驻点)及导数不存在但函数有意义的点(称这些点为极值疑点);

(3)把全部上面列出的点按大小列在表上,它们把定义域分割成若干区间,分别根据每个区间上导数的符号判断其单调性。

\textbf{2.曲线的凹凸性与拐点}
\begin{definition}[凹凸性的定义] \label{def:concave_up_down}
    若曲线弧上每一点的切线都位于曲线的下方,则称这段弧是凹的,若曲线弧上每一点的切线都位于曲线的上方,则称这段弧是凸的。
\end{definition}

\textbf{曲线的凹凸性判别法} \quad 设函数$f(x)$在区间$[a,b]$上连续,在区间$(a,b)$内具有二阶导数。如果$f''(x)\leq0$,但$f''(x)$在任何子区间中不恒为零,则曲线弧$y=f(x)$是凸的;如果$f''(x)\geq0$,但$f''(x)$在任何子区间内不恒为零,则曲线弧$y=f(x)$是凹的。

\begin{definition}[拐点的定义] \label{def:inflection_point}
    连续曲线凹与凸部分的分界点称为曲线的拐点。
\end{definition}

因为拐点是曲线凹凸弧的分界点,所以在拐点横坐标左右两侧邻近处$f''(x)$必然异号,而在拐点横坐标处$f''(x)$等于零或不存在。

\textbf{拐点存在的必要条件} \quad 设函数$f(x)$在$x_0$点具有二阶导数,则点$(x_0,f(x_0))$是曲线$y=f(x)$的拐点的必要条件是$f''(x_0)=0$。

\textbf{判定曲线凹凸性或求函数的凹、凸区间、拐点的步骤是:}

(1)求出函数的定义域或指定区域,及二阶导数;

(2)在区域内求出所有拐点疑点(二阶导数为0的点、二阶导数不存在但函数有意义的点),函数边界点及使函数无意义的端点,把这些点列在表上,根据二阶导数在各区间上的正负进行判断。

\section{函数的极值与最大值、最小值}
\textbf{1.函数的极值}
\begin{definition}[极值的定义] \label{def:extremum}
    设函数$f(x)$在$x_0$点的某个邻域内有定义,对于该邻域内异于$x_0$的点$x$,如果恒有$f(x)<f(x_0)$,则称$f(x_0)$为$f(x)$的极大值,而称$x_0$为$f(x)$的极大值点;如果恒有$f(x)>f(x_0)$,则称$f(x_0)$为$f(x)$的极小值,而称$x_0$为$f(x)$的极小值点。

    极大值与极小值统称为极值,极大值点与极小值点统称为极值点。
\end{definition}

\textbf{极值的必要条件} \quad 设函数$f(x)$在$x_0$点可导,且在$x_0$点取得极值,则必有$f'(x_0)=0$。

\textbf{极值第一判别法} \quad 设函数$f(x)$在$x_0$点的某个邻域内可导,且$f'(x_0)=0$,那么

(1)若当$x<x_0$时,$f'(x)>0$;当$x>x_0$时$f'(x)<0$,则$f(x_0)$是$f(x)$的极大值。

(2)若当$x<x_0$时,$f'(x)<0$;当$x>x_0$时$f'(x)>0$,则$f(x_0)$是$f(x)$的极小值。

(3)若在$x_0$的两侧,$f'(x)$的符号相同,则$f(x_0)$不是极值。

\textbf{极值第二判别法} \quad 设函数$f(x)$在$x_0$点处有二阶导数,且$f'(x_0)=0,f''(x_0)\neq0$,则

(1)当$f''(x_0)<0$时,函数$f(x)$在点$x_0$取得极大值;

(2)当$f''(x_0)>0$时,函数$f(x)$在点$x_0$取得极小值。

\textbf{2.求极值的步骤}

(1)求出函数$f(x)$的全部极值疑点——驻点($f'(x)=0$的点)及导数不存在但函数有意义的内点;

(2)逐个地进行判断。判断的方法一般有两个

方法一:用第一种充分条件,求出导函数$f'(x)$并把它因式分解,根据极值疑点邻近$f'(x)$的符号判断。如果极值疑点较多时,亦可先列表求出单调区间,然后根据各单调区间进行判断。

方法二:用第二种充分条件,即如果是驻点,用二阶导数在该点处的正负判断。

\begin{note}
    注意方法二的条件是极值疑点必为驻点;该点处存在二阶导数且不为0,否则应改用方法一判断。当$f''(x)$存在但较复杂时,一般也用方法一判断。
\end{note}

\textbf{3.函数的最大值与最小值}

设函数$f(x)$在$[a,b]$上连续,在$(a,b)$内仅有一个极值点,则若$x_0$是$f(x)$的极大值点,那么$x_0$必为$f(x)$在$[a,b]$上的最大值点;若$x_0$是$f(x)$的极小值点,那么$x_0$必为$f(x)$在$[a,b]$上的最小值点。

\textbf{4.求函数最值的步骤}

(1)找出此区间上的全部极值疑点(即驻点、导数不存在但函数有意义的内点)及使函数有定义的边界点;

(2)分别求出函数在这些点上的函数值并比较其大小,其中最大的函数值就是最大值,最小的函数值就是最小值。注意,若函数在指定区间单调且在边界点处连续,则其边界点必为最值点。

\section{函数图形的描绘}
\textbf{曲线的渐近线}

若$\displaystyle \lim_{x\rightarrow+\infty}f(x)=A$,则称直线$y=A$为曲线$y=f(x)$的水平渐近线(将$x\rightarrow+\infty$改为$x\rightarrow-\infty$仍有此定义).

若$\displaystyle \lim_{x\rightarrow x_0^+}f(x)=\infty$,则称直线$x=x_0$为曲线$y=f(x)$的铅直渐近线(将$x\rightarrow x_0^+$改为$x\rightarrow x_0^-$仍有此定义).

若$\displaystyle \lim_{x\rightarrow+\infty}\dfrac{f(x)}{x}=a(a\neq0)$,且$\displaystyle \lim_{x\rightarrow+\infty}[f(x)-ax]=b$,则称直线$y=ax+b$为曲线$y=f(x)$的斜渐近线(将$x\rightarrow+\infty$改为$x\rightarrow-\infty$仍有此定义).

\textbf{作图步骤}

(1)写出函数$f(x)$,标出定义域或指定的作图区域;

(2)判断$f(x)$的奇偶性、周期性,(如果有这样的特性,可以缩小作图范围);

(3)求水平渐近线、铅直渐近线与斜渐近线;

(4)求出$f'(x),f''(x)$,从而求出作图的关键点:极值疑点,拐点疑点,函数$f(x)$的边界点及无意义端点;

(5)列表;

(6)作图,如果作图的关键点(无意义点除外)不够,还可多描一些点,例如$f(x)$与坐标轴的交点等。

\section{曲率}
\begin{definition}[曲率的定义] \label{def:curvature}
    在曲线$L$上,有点$N$沿曲线$L$趋于点$M$时,如果极限$\displaystyle\lim_{\Delta s\rightarrow0}\bar{K}=\lim_{\Delta s\rightarrow0}\left|\dfrac{\Delta\alpha}{\Delta s}\right|$存在,则称此极限值为曲线$L$在点$M$处的曲率,记作$\displaystyle K=\lim_{\Delta s\rightarrow0}\left|\dfrac{\Delta \alpha}{\Delta s}\right|$。在$\displaystyle\lim_{\Delta s\rightarrow 0}\dfrac{\Delta \alpha}{\Delta s} = \dfrac{\deriv \alpha}{\deriv s}$存在的条件下,$K$也可以表示为$K=\left|\dfrac{\deriv \alpha}{\deriv s}\right|$.
\end{definition}

\textbf{计算曲率的公式}

设曲线的直角坐标方程是$y=f(x)$,且$f(x)$具有二阶导数,则得曲率公式
\begin{equation}
    K=\dfrac{\left|y''\right|}{(1+y'^2)^{3/2}}
    \nonumber
\end{equation}

若曲线由参数方程$\displaystyle \left\{\begin{array}{l} x=\varphi(t) \\ y=\psi(t) \end{array}\right.$$(\alpha\leq t \leq\beta)$给出,则可利用由参数方程确定的函数的求导法,求出$y'_x$及$y''_x$,代入曲率公式即可。