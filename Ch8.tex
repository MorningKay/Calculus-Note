\setcounter{chapter}{7}

\chapter{多元函数微分法及其应用}

\begin{introduction}
    \item 多元函数基本概念
    \item 偏导数与全微分
    \item 多元复合函数的求导法则
    \item 隐函数的求导法则
    \item 多元函数微分学的几何应用
    \item 方向导数与梯度
    \item 多元函数的极值及其求法
\end{introduction}

\section{多元函数的基本概念}
\textbf{1.二元函数的概念}

设有变量$x,y$和$z$,如果变量$x,y$在一定范围内取定一组值时,变量$z$按照一定的法则,总有唯一确定的数值与之对应,则称$z$是$x,y$的二元函数,记为
\begin{equation*}
    z=f(x,y)
\end{equation*}
并称$x,y$为自变量.

自变量$x,y$的取值范围,叫做函数的定义域.

在空间直角坐标系中,二元函数$z=f(x,y)$的图形通常是一张曲面,它的定义域是这张曲面在$xOy$平面上的投影.

类似地,可以定义三元以及三元以上的函数。二元及二元以上的函数,统称多元函数.

\textbf{2.二元函数的极限}

设二元函数$z=f(x,y)$定义在平面点集$E$上,$P_0(x_0,y_0)$是$E$的聚点,$A$为一常数。若对于任意给定的正数$\varepsilon$,总存在正数$\delta$,使得适合不等式$0<|P_0P|=\sqrt{(x-x_0)^2+(y-y_0)^2}<\delta$的一切点$P(x,y)$,都有
\begin{equation*}
    |f(x,y)-A|<\varepsilon
\end{equation*}
成立,则称$A$为函数$z=f(x,y)$当$x\rightarrow x_0, y\rightarrow y_0$时的极限,记为$\displaystyle \lim_{x\rightarrow x_0 \atop y\rightarrow y_0}f(x,y)=A$。这时也称当$x\rightarrow x_0, y\rightarrow y_0$时,函数$f(x,y)$收敛于$A$.

为了区别于一元函数极限,把上述二元函数的极限叫做二重极限.

所谓二重极限存在,是指点$P(x,y)$以任何方式无限趋于点$P_0(x_0,y_0)$时,函数$f(x,y)$都趋于同一数值$A$。因此,如果点$P(x,y)$以某一特殊方式,例如沿某一定直线或定曲线趋近于$P_0(x_0,y_0)$时,即使函数趋于某一确定值,也不能由此断定函数的极限存在。但是反过来,如果当$P(x,y)$以不同方式趋于$P_0(x_0,y_0)$时,函数趋于不同的值,那么就可以断定该函数的极限不存在.

\textbf{3.二元函数的连续性}

设函数$z=f(x,y)$的定义域为$D$,$P_0(x_0,y_0)$是$D$内的聚点,且$P_0\in D$,若
\begin{equation*}
    \lim_{x\rightarrow x_0 \atop y\rightarrow y_0}f(x,y)=f(x_0,y_0)
\end{equation*}
则称函数$z=f(x,y)$在点$P_0$处连续.

若函数在区域$D$内的每一点都连续,则称函数$z=f(x,y)$在区域$D$内连续.

多元初等函数在其定义域内是连续函数.

\textbf{4.有界闭区域上二元连续函数的性质}

\textbf{最大值和最小值定理} \quad 在有界闭区域上的二元连续函数,在该区域上至少取得它的最大值和最小值各一次.

\textbf{介值定理} \quad 在有界闭区域上的二元连续函数,如果取得两个不同的函数值,则函数在该区域上必取得介于这两个值之间的任何值.

特别地,若$\mu$是介于在有界闭区域上连续的函数$f(x,y)$的最小值$m$和最大值$M$之间的一个数,则在该区域中至少存在一点$P(\xi, \eta)$,使得$f(\xi, \eta)=\mu$.

\section{偏导数}
\textbf{1.偏导数定义}
\begin{align*}
    & \dfrac{\partial z}{\partial x}=\lim_{\Delta x\rightarrow 0}\dfrac{f(x+\Delta x,y)-f(x,y)}{\Delta x} \\
    & \dfrac{\partial z}{\partial y}=\lim_{\Delta y\rightarrow 0}\dfrac{f(x,y+\Delta y)-f(x,y)}{\Delta y}
\end{align*}

\textbf{2.高阶偏导数}

函数$z=f(x,y)$在区域$D$内的偏导数$f_x'(x,y),f_y'(x,y)$存在时,仍然是$x,y$的二元函数。若这两个函数的偏导数
\begin{align*}
    & \dfrac{\partial}{\partial x}\left(\dfrac{\partial z}{\partial x}\right)=\dfrac{\partial^2 z}{\partial x^2}=f_{xx}''(x,y) \\
    & \dfrac{\partial}{\partial y}\left(\dfrac{\partial z}{\partial x}\right)=\dfrac{\partial^2 z}{\partial x\partial y}=f_{xy}''(x,y) \\
    & \dfrac{\partial}{\partial x}\left(\dfrac{\partial z}{\partial y}\right)=\dfrac{\partial^2 z}{\partial y\partial x}=f_{yx}''(x,y) \\
    & \dfrac{\partial}{\partial y}\left(\dfrac{\partial z}{\partial y}\right)=\dfrac{\partial^2 z}{\partial y^2}=f_{yy}''(x,y)
\end{align*}
也存在,则称它们是函数$z=f(x,y)$的二阶偏导数.

二阶偏导数$\dfrac{\partial^2 z}{\partial x\partial y}$和$\dfrac{\partial^2 z}{\partial y\partial x}$称为函数$z=f(x,y)$的二阶混合偏导数。当这两个二阶混合偏导数在区域$D$内连续时,则在该区域$D$内有
\begin{equation*}
    \dfrac{\partial^2 z}{\partial x\partial y}=\dfrac{\partial^2 z}{\partial y\partial x}
\end{equation*}

\section{全微分}
\textbf{1.全微分的定义}

设$P_0(x_0,y_0)$为$f$定义域$D$内的一个内点,如果函数$z=f(x,y)$在点$P_0(x_0,y_0)$处的全增量$\Delta z$可表示为
\begin{equation*}
    \Delta z=f(x_0+\Delta x,y_0+\Delta y)-f(x_0,y_0)=A\cdot\Delta x+B\cdot\Delta y+o(\rho)
\end{equation*}
其中$A, B$是与$\Delta x, \Delta y$无关的常数。则称函数$z=f(x,y)$在点$P_0(x_0,y_0)$处可微,并称函数$z=f(x,y)$的全增量$\Delta z$的线性主部$A\cdot\Delta x+B\cdot\Delta y$为函数$z=f(x,y)$在点$P_0(x_0,y_0)$的全微分,记作
\begin{equation*}
    \deriv z=A\cdot\Delta x+B\cdot\Delta y=A\deriv x+B\deriv y \quad (\deriv x=\Delta x, \deriv y=\Delta y)
\end{equation*}
当函数$z=f(x,y)$在点$P_0(x_0,y_0)$处可微时有
\begin{equation*}
    \deriv z=f_x'(x_0,y_0)\deriv x+f_y'(x_0,y_0)\deriv y
\end{equation*}

\textbf{2.全微分的形式不变性}

设$z=f(u,v)$具有连续偏导数,$u=\varphi(x,y),v=\psi(x,y)$也具有连续偏导数,则复合函数$z=f[\varphi(x,y),\psi(x,y)]$在点$(x,y)$处的全微分为
\begin{equation*}
    \deriv z=\dfrac{\partial z}{\partial u}\deriv u+\dfrac{\partial z}{\partial v}\deriv v
\end{equation*}

\section{多元复合函数的求导法则}
\textbf{1.复合函数的偏导数}

设函数$u=\varphi(x,y),v=\psi(x,y)$在点$(x,y)$处存在偏导数,又函数$z=f(u,v)$在对应点$(u,v)$处具有连续的一阶偏导数,则复合函数$z=f[\varphi(x,y),\psi(x,y)]$在点$(x,y)$处对$x$及$y$的偏导数均存在,且有
\begin{equation*}
    \dfrac{\partial z}{\partial x}=\dfrac{\partial z}{\partial u}\cdot\dfrac{\partial u}{\partial x}+\dfrac{\partial z}{\partial v}\cdot\dfrac{\partial v}{\partial x}, \quad
    \dfrac{\partial z}{\partial y}=\dfrac{\partial z}{\partial u}\cdot\dfrac{\partial u}{\partial y}+\dfrac{\partial z}{\partial v}\cdot\dfrac{\partial v}{\partial y}
\end{equation*}

\textbf{2.全导数}
设函数$z=f(u,v)$,而$u=\varphi(x),v=\psi(x)$,则$z=f[\varphi(x),\psi(x)]$是$x$的一元函数,且
\begin{equation*}
    \dfrac{\deriv z}{\deriv x}=\dfrac{\partial z}{\partial u}\cdot\dfrac{\deriv u}{\deriv x}+\dfrac{\partial z}{\partial v}\cdot\dfrac{\deriv v}{\deriv x}
\end{equation*}
称$\dfrac{\deriv z}{\deriv x}$为$z$关于$x$的全导数。

\section{隐函数的求导法则}
\textbf{1.一元隐函数求导法则}

设函数$F(x,y)$在点$P(x_0,y_0)$的某个邻域内具有连续的偏导数$F_x'(x,y),F_y'(x,y)$,且$F(x_0,y_0)=0,F_y'(x_0,y_0)\neq 0$,则在$(x_0,y_0)$的某邻域内,方程$F(x,y)=0$恒能唯一确定一个具有连续导数的函数$y=f(x)$,它满足条件$y_0=f(x_0)$,并有
\begin{equation*}
    \dfrac{\deriv y}{\deriv x}=-\dfrac{F_x'(x,y)}{F_y'(x,y)}
\end{equation*}

\textbf{2.二元隐函数求导法则}

设函数$F(x,y,z)$在点$P(x_0,y_0,z_0)$的某个邻域内具有连续的偏导数$F_x'(x,y,z),F_y'(x,y,z),F_z'(x,y,z)$,且\\$F(x_0,y_0,z_0)=0,F_z'(x_0,y_0,z_0)\neq 0$,则在点$(x_0,y_0,z_0)$的某一邻域内,方程$F(x,y,z)=0$恒能唯一确定一个具有连续偏导数的函数$z=f(x,y)$,它满足条件$z_0=f(x_0,y_0)$,并有
\begin{equation*}
    \dfrac{\partial z}{\partial x}=-\dfrac{F_x'(x,y,z)}{F_z'(x,y,z)}, \quad
    \dfrac{\partial z}{\partial y}=-\dfrac{F_y'(x,y,z)}{F_z'(x,y,z)}
\end{equation*}

\section{多元函数微分学的几何应用}
\textbf{1.空间曲线的切线与法平面}

设空间曲线的参数方程为 \quad $\left\{\begin{aligned} & x = x(t) \\ & y = y(t) \\ & z = z(t) \end{aligned}\right.$
其中$x=x(t),y=y(t),z=z(t)$均为$t$的可微函数,且$x'(t),y'(t),z'(t)$不同时为零,则当$t=t_0$时,曲线上对应点$M_0(x_0,y_0,z_0)$处的切线方程为
\begin{equation*}
    \dfrac{x-x_0}{x'(t_0)}=\dfrac{y-y_0}{y'(t_0)}=\dfrac{z-z_0}{z'(t_0)}
\end{equation*}
法平面方程为
\begin{equation*}
    x'(t_0)(x-x_0)+y'(t_0)(y-y_0)+z'(t_0)(z-z_0)=0
\end{equation*}

\textbf{2.曲面的切平面与法线}

设曲面方程为$F(x,y,z)=0$,其中$F(x,y,z)$具有连续的偏导数$F_x',F_y',F_z'$,且它们不同时为零,则在曲面上点$M_0(x_0,y_0,z_0)$处的切平面方程为
\begin{equation*}
    F_x'(x_0,y_0,z_0)(x-x_0)+F_y'(x_0,y_0,z_0)(y-y_0)+F_z'(x_0,y_0,z_0)(z-z_0)=0
\end{equation*}
法线方程为
\begin{equation*}
    \dfrac{x-x_0}{F_x'(x_0,y_0,z_0)}=\dfrac{y-y_0}{F_y'(x_0,y_0,z_0)}=\dfrac{z-z_0}{F_z'(x_0,y_0,z_0)}
\end{equation*}

若曲面方程为$z=f(x,y)$,且$f(x,y)$具有连续的偏导数,则曲面上点$M_0(x_0,y_0,z_0)$处切平面方程为
\begin{equation*}
    f_x'(x_0,y_0)(x-x_0)+f_y'(x_0,y_0)(y-y_0)-(z-z_0)=0
\end{equation*}
法线方程为
\begin{equation*}
    \dfrac{x-x_0}{f_x'(x_0,y_0)}=\dfrac{y-y_0}{f_y'(x_0,y_0)}=\dfrac{z-z_0}{-1}
\end{equation*}

\section{方向导数与梯度}
\textbf{1.方向导数}

\begin{wrapfigure}{r}[0cm]{0pt}   
    \begin{tikzpicture}[scale=0.6]
        \node[below=0.2cm,left] (o) at (0,0) {$O$};
        %axis
        \draw[->] (-1,0) -- (6,0) node[right] {$x$};
        \draw[->] (0,-1) -- (0,6) node[above] {$y$};
        % point
        \coordinate (p) at (1,1);
        \coordinate (pp) at (4,4);
        \coordinate (a) at (4,1);
        % node
        \node[below,left] at (p) {$P$};
        \node[above] (pp) at (pp) {$P'$};
        % line
        \draw[->] (1,1) -- node[pos=0.5, above] {$\rho$} (4,4) -- (5,5) node[above right] {$l$};
        \draw[dashed] (1,1) -- node[below] {$\Delta x$} (a) -- node[right] {$\Delta y$} (pp);
        \draw pic[draw, ->, angle radius=0.7cm, "$\alpha$" shift={(6mm,1mm)}] {angle = a--p--pp};
    \end{tikzpicture}
    \caption{} \label{fig:8-7-1}
\end{wrapfigure}
设函数$z=f(x,y)$在点$P(x,y)$的某个邻域内有定义,过点$P$引射线$l$(如图\ref{fig:8-7-1}),在$l$上点$P$的邻近取一动点
\begin{equation*}
    P'(x+\Delta x, y+\Delta y)
\end{equation*}
记$P$与$P'$的距离为
\begin{equation*}
    \rho=\sqrt{\Delta x^2+\Delta y^2}
\end{equation*}
当$P'$沿$l$趋于$P$时,如果极限
\begin{equation*}
    \lim_{P'\rightarrow P}\dfrac{f(P')-f(P)}{|PP'|}=\lim_{\rho\rightarrow 0}\dfrac{f(x+\Delta x,y+\Delta y)-f(x,y)}{\rho}
\end{equation*}
存在,则称此极限值为函数$z=f(x,y)$在点$P$沿方向$l$的方向导数,记为$\dfrac{\partial z}{\partial l}$.


当函数$z=f(x,y)$在点$P(x,y)$处可微,射线$l$的方向余弦为$\cos\alpha,\cos\beta$时
\begin{equation*}
    \dfrac{\partial z}{\partial l}=\dfrac{\partial z}{\partial x}\cdot\cos\alpha+\dfrac{\partial z}{\partial y}\cdot\cos\beta
\end{equation*}

同样,三元函数$u=f(x,y,z)$在点$P(x,y,z)$处可微时,则沿方向余弦为$\cos\alpha,\cos\beta,\cos\gamma$的射线$l$的方向导数为
\begin{equation*}
    \dfrac{\partial u}{\partial l}=\dfrac{\partial u}{\partial x}\cdot\cos\alpha+\dfrac{\partial u}{\partial y}\cdot\cos\beta+\dfrac{\partial u}{\partial z}\cdot\cos\gamma
\end{equation*}

\textbf{2.梯度}

设函数$z=f(x,y)$具有连续的一阶偏导数,则函数$z$在$P(x,y)$处的梯度是一个向量,记为$\text{grad}z$,它在$x,y$坐标轴上的投影分别为在该点处的偏导数$\dfrac{\partial z}{\partial x}$和$\dfrac{\partial z}{\partial y}$,即
\begin{equation*}
    \text{grad}z=\dfrac{\partial z}{\partial x}\bm{i}+\dfrac{\partial z}{\partial y}\bm{j}
\end{equation*}

函数$z=f(x,y)$在点$P(x,y)$处沿$l$方向上的方向导数$\dfrac{\partial z}{\partial l}$,等于函数在该点处的梯度$\text{grad}z$在$l$方向上的投影,即
\begin{equation*}
    \dfrac{\partial z}{\partial l}=\text{grad}z\cdot\bm{l}^{\circ}
\end{equation*}
其中,$\bm{l}^{\circ}$是射线$l$方向上的单位向量.

函数$z=f(x,y)$在点$P$处的梯度$\text{grad}z$的模是函数$z$在该点处方向导数的最大值,它的方向与函数$z$在点$P$处取得最大方向导数的方向一致.

同样,三元函数$u=f(x,y,z)$具有连续的一阶偏导数时,函数$u$在点$P(x,y,z)$处的梯度为
\begin{equation*}
    \text{grad}u=\dfrac{\partial u}{\partial x}\bm{i}+\dfrac{\partial u}{\partial y}\bm{j}+\dfrac{\partial u}{\partial z}\bm{k}
\end{equation*}

\section{多元函数的极值及其求法}

\textbf{1.极值}

(1)\textbf{极值的定义} \quad 设函数$z=f(x,y)$在点$P_0(x_0,y_0)$的某个邻域内有定义,对于该邻域内异于$P_0(x_0,y_0)$的点$P(x,y)$,如果都满足不等式$f(x,y)<f(x_0,y_0)$,则称函数在点$P_0(x_0,y_0)$处有极大值$f(x_0,y_0)$;如果都满足不等式$f(x,y)>f(x_0,y_0)$,则称函数在点$P_0(x_0,y_0)$处有极小值$f(x_0,y_0)$。极大值、极小值统称为极值,使函数取得极值的点称为极值点.

(2)\textbf{极值存在的必要条件} 

若函数$z=f(x,y)$在点$P_0(x_0,y_0)$处可微且取得极值,则必有$f_x'(x_0,y_0)=0,f_y'(x_0,y_0)=0$.

(3)\textbf{极值存在的充分条件} 

设函数$z=f(x,y)$在点$P_0(x_0,y_0)$的某邻域内具有连续二阶偏导数,且$f_x'(x_0,y_0)=0,f_y'(x_0,y_0)=0$,记$A=f_{xx}''(x_0,y_0),B=f_{xy}''(x_0,y_0),C=f_{yy}''(x_0,y_0)$,则

1)若$B^2-AC<0$,则$f(x_0,y_0)$是极值,当$A<0$时,$f(x_0,y_0)$是极大值,当$A>0$时,$f(x_0,y_0)$是极小值.

2)若$B^2-AC>0$,则$f(x_0,y_0)$不是极值.

3)若$B^2-AC=0$,则$f(x_0,y_0)$不是极值.

\textbf{2.条件极值、拉格朗日乘数法}

函数$u=f(x,y)$在附加条件$\varphi(x,y)=0$下的极值称为条件极值

\textbf{拉格朗日乘数法} \quad 求条件极值时,可作函数
\begin{equation*}
    F(x,y,\lambda)=f(x,y)+\lambda\varphi(x,y)
\end{equation*}
其中,$\lambda$是某一常数,则点$(x,y)$是极值点的必要条件为
\begin{equation*}
    \begin{cases}
        F_x'(x,y)=f_x'(x,y)+\lambda\varphi_x'(x,y)=0\\
        F_y'(x,y)=f_y'(x,y)+\lambda\varphi_y'(x,y)=0\\
        \varphi(x,y)=0
    \end{cases}
\end{equation*}
从上述方程组中解出$x,y$及$\lambda$的值,则点$(x,y)$就可能是条件极值的极值点.

\textbf{3.函数的最大值和最小值}

若二元函数$f(x,y)$在有界闭域$D$上连续,则$f(x,y)$在$D$上必能取得最大值和最小值.

求函数最大值、最小值的一般方法是把函数$f(x,y)$在区域$D$内部的所有可能的极值点处的函数值连同边界上的函数值加以比较,最大者为最大值,最小者为最小值.

如果根据实际问题的性质已经知道函数的最大值(最小值)一定在区域$D$内部取得,而函数在区域$D$内只有唯一驻点,则该驻点处的函数值就是函数$f(x,y)$在区域$D$上的最大值(最小值).