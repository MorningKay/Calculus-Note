\setcounter{chapter}{10}

\chapter{无穷级数}

\begin{introduction}
    \item 常数项级数
    \item 正项级数的审敛法
    \item 任意项级数的审敛法
    \item 幂级数
    \item 傅立叶级数
\end{introduction}

\section{常数项级数的概念和性质}

\textbf{常数项级数的概念}

设有数列$\{u_n\}:u_1,u_2,\cdots,u_n,\cdots$,将其各项累加的所得的式子$u_1+u_2+\cdots+u_n+\cdots$称为(常数项)无穷级数,简称(常数项)级数,记作$\displaystyle\sum_{n=1}^\infty u_n$,即$\displaystyle\sum_{n=1}^\infty u_n=u_1+u_2+\cdots+u_n+\cdots$。

\textbf{常数项级数收敛的概念}

设给定常数项级数
\begin{equation}
    \sum_{n=1}^\infty u_n=u_1+u_2+\cdots+u_n+\cdots
    \label{constant_series}
\end{equation}
记$S_n=\displaystyle\sum_{n=1}^\infty u_n=u_1+u_2+\cdots+u_n$为级数$\sum_{n=1}^\infty u_n$的前$n$项部分和,若$\displaystyle\lim_{n\rightarrow\infty}S_n=S$(有限值),则称级数(\ref{constant_series})收敛;若$\displaystyle\lim_{n\rightarrow\infty}S_n$不存在,则称级数(\ref{constant_series})发散。

\begin{property}
    \begin{enumerate}
        \item 级数$\displaystyle\sum_{n=1}^\infty u_n$与$\displaystyle\sum_{n=1}^\infty ku_n$有相同敛散性($k$是不为零的常数);
        
        \item 若级数$\displaystyle\sum_{n=1}^\infty u_n$,$\sum_{n=1}^\infty v_n$均收敛,则级数$\displaystyle\sum_{n=1}^\infty(u_n\pm v_n)$也收敛,且有
        \begin{equation*}
            \sum_{n=1}^\infty(u_n+v_n)=\sum_{n=1}^\infty u_n+\sum_{n=1}^\infty v_n
        \end{equation*}

        \item 在级数中去掉或添加有限项,不会影响级数的敛散性(但收敛时,级数和一般会改变);
        
        \item 收敛级数任意加括号后所成的级数仍收敛。如果正项级数加括号后所成的级数收敛,则原级数收敛;
        
        \item 级数收敛的必要条件:若$\displaystyle\sum_{n=1}^\infty u_n$收敛,则$\displaystyle\lim_{n\rightarrow\infty}u_n=0$.
    \end{enumerate}
\end{property}

\textbf{柯西收敛准则}

级数$\sum_{n=1}^\infty u_n$收敛的充分必要条件:对任意给定的正数$\varepsilon$,总存在$N$,使得当$n>N$时,对于任意的自然数$p=1,2,3,\cdots$,不等式
\begin{equation*}
    |u_{n+1}+u_{n+2}+\cdots+u_{n+p}|<\varepsilon
\end{equation*}
恒成立.

\section{正项级数的审敛法}
设$\displaystyle\sum_{n=1}^\infty u_n$,$\displaystyle\sum_{n=1}^\infty v_n$均为正项级数.

\textbf{1.比较审敛法}

(1) 若$0\leq u_n\leq v_n$,$\displaystyle\sum_{n=1}^\infty v_n$收敛,则$\displaystyle\sum_{n=1}^\infty u_n$收敛;

(2) 若$0\leq v_n\leq u_n$,$\displaystyle\sum_{n=1}^\infty v_n$发散,则$\displaystyle\sum_{n=1}^\infty u_n$发散.

比较审敛法的极限形式 \quad 若$\displaystyle\lim_{n\rightarrow\infty}\frac{u_n}{v_n}=\lambda(0<\lambda<+\infty)$,则级数$\displaystyle\sum_{n=1}^\infty u_n$与$\displaystyle\sum_{n=1}^\infty v_n$具有相同的敛散性.

\textbf{2.比值审敛法}

\begin{equation*}
    \lim_{n\rightarrow\infty}\frac{u_{n+1}}{u_n}=\rho
    \begin{cases}
        <1, \quad \mbox{则}\displaystyle\sum_{n=1}^\infty u_n\mbox{收敛}\\
        >1, \quad \mbox{则}\displaystyle\sum_{n=1}^\infty u_n\mbox{发散}\\
        =1, \quad \mbox{则}\displaystyle\sum_{n=1}^\infty u_n\mbox{敛散性不定}
    \end{cases}
\end{equation*}

\textbf{3.根植审敛法}

\begin{equation*}
    \lim_{n\rightarrow\infty}\sqrt[n]{u_n}=\rho
    \begin{cases}
        <1, \quad \mbox{则}\displaystyle\sum_{n=1}^\infty u_n\mbox{收敛}\\
        >1, \quad \mbox{则}\displaystyle\sum_{n=1}^\infty u_n\mbox{发散}\\
        =1, \quad \mbox{则}\displaystyle\sum_{n=1}^\infty u_n\mbox{敛散性不定}
    \end{cases}
\end{equation*}

\textbf{4.对数审敛法}

(1) 若存在$\alpha>0$,使当$n\geq n_0$时,$\dfrac{\ln\dfrac{1}{u_n}}{\ln n}\geq 1+\alpha$,则正项级数$\displaystyle\sum_{n=1}^\infty u_n$收敛;

(2) 若$n\geq n_0$时,$\dfrac{\ln\dfrac{1}{u_n}}{\ln n}\leq 1$,则正项级数$\displaystyle\sum_{n=1}^\infty u_n$发散.

\textbf{5.两个重要级数的敛散性}

等比级数:$\displaystyle\sum_{n=1}^\infty ar^n(a\neq0)$,当$|r|<1$时收敛,当$|r|\geq 1$时发散.

$p$-级数:$\displaystyle\sum_{n=1}^\infty \frac{1}{n^p}$,当$p>1$时收敛,当$p\leq 1$时发散.

\textbf{6.正项级数$\displaystyle\sum_{n=1}^\infty u_n$判断敛散性的一般步骤}

(1) 考察$u_n\xrightarrow{?}0$,若$\displaystyle\lim_{n\rightarrow\infty}u_n\neq 0$,则级数发散;

(2) 若$u_n\rightarrow0$,用比值法或根值法判定级数敛散性;

(3) 若比值法或根值判别法均无效,则用比较判别法;

(4) 若上述方法都行不通时,考虑$S_n$是否有极限

从上述步骤可知,比值法或根值法是比较重要的判别法,也是比较易掌握的判别法

\section{任意项级数的审敛法}

\textbf{1.交错级数的莱布尼兹判别法}

若$u_n>0,u_n\geq u_{n+1},\displaystyle\lim_{n\rightarrow\infty}u_n=0$,则交错级数$\displaystyle\sum_{n=1}^\infty(-1)^{n-1}u_n$收敛,其和$S<u_1$.

\textbf{2.任意项级数 \quad 绝对收敛与条件收敛}

若$\displaystyle\sum_{n=1}^\infty u_n$为任意项级数,且$\displaystyle\sum_{n=1}^\infty|u_n|$收敛,则$\displaystyle\sum_{n=1}^\infty u_n$收敛,并称$\displaystyle\sum_{n=1}^\infty u_n$为绝对收敛;若$\displaystyle\sum_{n=1}^\infty u_n$收敛,而$\displaystyle\sum_{n=1}^\infty|u_n|$发散,则称$\displaystyle\sum_{n=1}^\infty u_n$为条件收敛.

\textbf{3.判定任意项级数$\displaystyle\sum_{n=1}^\infty u_n$的敛散性的主要方法}

若$\displaystyle\sum_{n=1}^\infty |u_n|$收敛,则$\displaystyle\sum_{n=1}^\infty u_n$绝对收敛;若$\displaystyle\sum_{n=1}^\infty |u_n|$发散,则$\displaystyle\sum_{n=1}^\infty u_n$敛散性判别主要利用“莱布尼兹判别法”或$u_n\rightarrow0$或求$S_n$.

\section{幂级数}
\textbf{1.函数项级数的一般概念}

(1)函数项级数的定义 \quad 设给定一个定义在区间$[a,b]$上的函数列
\begin{equation*}
    u_1(x),u_2(x),\cdots,u_n(x),\cdots,
\end{equation*}
则式子
\begin{equation}
    u_1(x)+u_2(x)+\cdots+u_n(x)+\cdots \label{function_series}
\end{equation}
叫做函数项级数

(2)函数项级数的收敛域 \quad 对于区间$[a,b]$上的每一个值$x_0$,级数(\ref{function_series})成为常数项级数
\begin{equation}
    u_1(x_0)+u_2(x_0)+\cdots+u_n(x_0)+\cdots \label{func_const_series}
\end{equation}
如果(\ref{func_const_series})收敛,则称$x_0$是级数(\ref{function_series})的收敛点;如果(\ref{func_const_series})发散,则称$x_0$是级数(\ref{function_series})的发散点,(\ref{function_series})的所有收敛点的全体称为函数项级数(\ref{function_series})的收敛域.

(3)函数项级数的和函数 \quad 对于收敛域内的任一点$x$,级数(\ref{function_series})都有一个确定的和
\begin{equation*}
    S(x)=\sum_{n=1}^\infty u_n(x)
\end{equation*}
$S(x)$是定义在收敛域上的函数,称为级数(\ref{function_series})的和函数.

\textbf{2.幂级数及其收敛域}

(1)幂级数的定义 \quad 形如
\begin{equation*}
    a_0+a_1(x-x_0)+a_2(x-x_0)^2+\cdots+a_n(x-x_0)^n+\cdots
\end{equation*}
或
\begin{equation*}
    a_0+a_1x+a_2x^2+\cdots+a_nx^n+\cdots
\end{equation*}
的级数称为幂级数.

(2)阿贝尔(Abel)引理 \quad 若$x_0$是幂级数$\displaystyle\sum_{n=0}^\infty a_n x^n$的收敛点,则对于一切满足$|x|<|x_0|$的点$x$,幂级数都绝对收敛;若$x_0$是幂级数$\displaystyle\sum_{n=0}^\infty a_n x^n$的发散点,则对于一切满足$|x|>|x_0|$的点$x$,幂级数都发散.

(3)幂级数的收敛半径 \quad 对任一幂级数$\displaystyle\sum_{n=0}^\infty a_n x^n$,必存在一个非负数$R$($R$可为无穷大),使得对一切$|x|<R$的点$x$(当$R=0$时,$x=0$),幂级数都收敛;而对一切$|x|>R$的点$x$,幂级数都发散,$R$称为幂级数$\displaystyle\sum_{n=0}^\infty a_n x^n$的收敛半径,$R$的求法如下:

设幂级数$\displaystyle\sum_{n=0}^\infty a_n x^n$,若$\displaystyle\lim_{n\rightarrow\infty}\left|\dfrac{a_{n+1}}{a_n}\right|=\rho$,则幂级数的收敛半径
\begin{equation*}
    R=\begin{cases}
        \dfrac{1}{\rho}, \quad 0<\rho<+\infty\\
        +\infty, \quad \rho=0\\
        0, \quad \rho=\infty
    \end{cases}
\end{equation*}

(4)幂级数的收敛域 \quad 在幂级数$\displaystyle\sum_{n=0}^\infty a_n x^n$的收敛区间$(-R,R)$上,加上收敛区间端点中的收敛点,就得到幂级数$\displaystyle\sum_{n=0}^\infty a_n x^n$的收敛域。若$R$是不为零的有限数,则其收敛域为以下四种情形之一:
\begin{equation*}
    (-R,R), \quad [-R,R], \quad (-R,R], \quad [-R,R)
\end{equation*}

\textbf{3.幂级数的性质}

若幂级数$\displaystyle\sum_{n=0}^\infty a_n x^n$的收敛半径为$R$,则有

(1)和函数$S(x)$在$(-R,R)$内是连续的,若$\displaystyle\sum_{n=0}^\infty a_n x^n$在端点$x=R$(或$x=-R$)处收敛,则和函数在点$x=R$左连续(或在点$x=-R$右连续).

(2)幂函数可以逐项微分,即
\begin{equation*}
    S'(x)=\left(\sum_{n=0}^\infty a_nx^n\right)'=\sum_{n=0}^\infty \left(a_nx^n\right)'=\sum_{n=0}^\infty na_nx^{n-1}, \quad x\in(-R,R)
\end{equation*}
若逐项微分后得到的幂级数$\displaystyle\sum_{n=0}^\infty na_nx^{n-1}$在端点$x=R$(或$x=-R$)处收敛,则逐项微分以前的幂级数在点$x=R$(或$x=-R$)处也收敛.

(3)幂级数可以逐项积分,即
\begin{equation*}
    \int_0^x S(x)\deriv x=\int_0^x \left(\sum_{n=0}^\infty a_nx^n\right)\deriv x=\sum_{n=0}^\infty \int_0^x a_nx^n\deriv x=\sum_{n=0}^\infty \dfrac{a_n}{n+1}x^{n+1}, \quad x\in(-R,R)
\end{equation*}
若幂级数$\displaystyle\sum_{n=0}^\infty a_nx^n$在端点$x=R$(或$x=-R$)处收敛,则积分上限$x$可取为$x=R$(或$x=-R$).

\textbf{4.幂级数的运算}

设$\displaystyle\sum_{n=0}^\infty a_nx^n=f(x)$的收敛半径为$R_1$,$\displaystyle\sum_{n=0}^\infty b_nx^n=h(x)$的收敛半径为$R_2$,则对于这两个幂级数可以进行下列四则运算:
\begin{equation*}
    \left(\sum_{n=0}^\infty a_nx^n\right)\pm\left(\sum_{n=0}^\infty b_nx^n\right)=\sum_{n=0}^\infty (a_n\pm b_n)x^n=f(x)\pm h(x)
\end{equation*}
收敛半径 \quad $R=\min\{R_1,R_2\}$
\begin{equation*}
    \left(\sum_{n=0}^\infty a_nx^n\right)\left(\sum_{n=0}^\infty b_nx^n\right)=\sum_{n=0}(a_0b_n+a_1b_{n-1}+\cdots+a_nb_0)x^n=f(x)\cdot h(x)
\end{equation*}
收敛半径 \quad $R=\min\{R_1,R_2\}$
\begin{equation*}
    \dfrac{\displaystyle\sum_{n=0}^\infty a_nx^n}{\displaystyle\sum_{n=0}^\infty b_nx^n}=\sum_{n=0}^\infty c_nx^n, \quad (\mbox{其中}b_0\neq0)
\end{equation*}

系数$c_n$可由幂级数的乘法$\displaystyle\left(\sum_{n=0}^\infty b_nx^n\right)\left(\sum_{n=0}^\infty c_nx^n\right)=\sum_{n=0}^\infty a_nx^n$,并比较同次幂的系数得到。相除后得到的幂级数$\displaystyle\sum_{n=0}^\infty c_nx^n$的收敛区间可能比原来两个幂级数的收敛区间小得多.

\section{函数展开成幂级数}
\textbf{1.泰勒(Taylor)级数}

当$f(x)$在$x_0$的某邻域内存在任意阶导数时,幂级数
\begin{align*}
    & \sum_{n=0}^\infty \dfrac{f^{(n)}(x_0)}{n!}(x-x_0)^n \\
    = & f(x_0)+f'(x_0)(x-x_0)+\dfrac{f''(x_0)}{2!}(x-x_0)^2+\cdots+\dfrac{f^{(n)}(x_0)}{n!}(x-x_0)^n+\cdots
\end{align*}
称为$f(x)$在点$x_0$处的泰勒级数.

当$x_0=0$时,泰勒级数为
\begin{equation*}
    \sum_{n=0}^\infty \dfrac{f^{(n)}(0)}{n!}x^n=f(0)+f'(0)x+\dfrac{f''(0)}{2!}x^2+\cdots+\dfrac{f^{(n)}(0)}{n!}x^n+\cdots
\end{equation*}
称为麦克劳林(Maclaurin)级数.

\textbf{2.函数的幂级数展开式}

函数展为幂级数有直接方法与间接方法两种(以下主要讨论函数$f(x)$在点$x=0$处展为幂级数的问题).

直接法 \quad 用直接法将函数展开为$x$的幂级数的步骤是

(1)求出$f(x)$在$x=0$处各阶导数值$f^{(n)}(0),n=0,1,2,\cdots
$

(2)写出幂级数
\begin{equation*}
    \sum_{n=0}^\infty \dfrac{f^{(n)}(0)}{n!}x^n=f(0)+f'(0)x+\dfrac{f''(0)}{2!}x^2+\cdots+\dfrac{f^{(n)}(0)}{n!}x^n+\cdots
\end{equation*}
并求出收敛半径$R$.

(3)在收敛区间$(-R,R)$内考察泰勒级数余项$R_n(x)$的极限
\begin{equation*}
    \lim_{n\rightarrow\infty}R_n(x)=\lim_{n\rightarrow\infty}\dfrac{f^{(n+1)}(\xi)}{(n+1)!}x^{n+1}\quad (\xi\mbox{介于}0\mbox{与}x\mbox{之间})
\end{equation*}
是否为零,如果为零,则第(2)步写出的幂级数就是$f(x)$的幂级数展开式.

间接法 \quad 这种方法是利用已知的函数展开式,经过适当的四则运算、复合步骤以及逐项微分、逐项积分等把所给函数展为幂级数。常用的函数展开式有
\begin{align*}
    & \dfrac{1}{1-x}=1+x+x^2+\cdots+x^n+\cdots \quad (-1,1) \\
    & e^x=1+x+\dfrac{x^2}{2!}+\cdots+\dfrac{x^n}{n!}+\cdots \quad (-\infty,\infty) \\
    & \sin x=x-\dfrac{x^3}{3!}+\dfrac{x^5}{5!}-\cdots+(-1)^{n-1}\dfrac{x^{2n-1}}{(2n-1)!}+\cdots \quad (-\infty,\infty) \\
    & \cos x=1-\dfrac{x^2}{2!}+\dfrac{x^4}{4!}-\cdots+(-1)^n\dfrac{x^{2n}}{(2n)!}+\cdots \quad (-\infty,\infty) \\
    & \ln (1+x)=x-\dfrac{x^2}{2}+\dfrac{x^3}{3}-\cdots+(-1)^{n-1}\dfrac{x^n}{n}+\cdots \quad (-1,1] \\
    & (1+x)^m=1+mx+\dfrac{m(m-1)}{2!}x^2+\cdots+\dfrac{m(m-1)\cdots(m-n+1)}{n!}x^n+\cdots \quad (-1,1)
\end{align*}
在最后一个式子中,当$x=\pm 1$时,级数是否收敛取决于$m$值.

\section{傅立叶级数}
\textbf{1.函数的傅立叶(Fourier)级数}

设函数$f(x)$在区间$[-\pi,\pi]$(或$[0,2\pi]$)上可积,则称
\begin{align*}
    & a_n=\dfrac{1}{\pi}\int_{-\pi}^\pi f(x)\cos nx\deriv x, \quad n=0,1,2,\cdots \\
    & b_n=\dfrac{1}{\pi}\int_{-\pi}^\pi f(x)\sin nx\deriv x, \quad n=1,2,3,\cdots
\end{align*}
为函数$f(x)$的傅立叶系数。由上述$a_n,b_n$所形成的三角级数
\begin{equation*}
    \dfrac{a_0}{2}+\sum_{n=1}^\infty(a_n\cos nx+b_n\sin nx)
\end{equation*}
称为函数$f(x)$的傅立叶级数.

\textbf{2.狄立克莱(Dirichlet)定理}

设函数$f(x)$在区间$[-\pi,\pi]$(或$[0,2\pi]$)上满足条件:

(1)连续或只有有限个第一类间断点;

(2)至多只有有限个极值点.\\
则$f(x)$的傅立叶级数
\begin{equation*}
    \dfrac{a_0}{2}+\sum_{n=1}^\infty(a_n\cos nx+b_n\sin nx)
\end{equation*}
在区间$[-\pi,\pi]$(或$[0,2\pi]$)上收敛,并且,若其和函数为$S(x)$,则有:

(1)在$f(x)$的连续点处,$S(x)=f(x)$;
\vspace{2mm}

(2)在$f(x)$的间断点$x$处,$S(x)=\dfrac{f(x-0)+f(x+0)}{2}$.
\vspace{2mm}

(3)在端点$x=\pm\pi$处,$S(x)=\dfrac{f(-\pi+0)+f(\pi-0)}{2}$.\\\vspace{2mm}
(或在$x=0,2\pi$处,$S(x)=\dfrac{f(0+0)+f(2\pi-0)}{2}$),其中$f(x_0-0),f(x_0+0)$分别表示$f(x)$在$x_0$处的左、右极限

\textbf{3.正弦级数}

若$f(x)$是$(-\pi,\pi)$上的奇函数,则
\begin{equation*}
    f(x)=\sum_{n=1}^\infty b_n\sin nx
\end{equation*}
其中$b_n=\displaystyle\dfrac{2}{\pi}\int_{-\pi}^\pi f(x)\sin nx\deriv x, \quad n=1,2,\cdots$.

\textbf{4.余弦级数}

若$f(x)$是$(-\pi,\pi)$上的偶函数,则
\begin{equation*}
    f(x)=\dfrac{a_0}{2}+\sum_{n=1}^\infty a_n\cos nx
\end{equation*}
其中$a_n=\displaystyle\dfrac{2}{\pi}\int_{-\pi}^\pi f(x)\cos nx\deriv x, \quad n=0,1,2,\cdots$.

\section{一般周期函数的傅立叶级数}
\textbf{任意区间$[-l,l]$上的傅立叶级数}

设函数$f(x)$在长为$2l$的区间$[-l,l]$(或$[0,2l]$)上满足狄立克莱定理条件,作变量置换$t=\dfrac{\pi x}{l}$,则函数$f(x)=f(\dfrac{l}{\pi}t)=\varphi(t)$,而$\varphi(t)$在区间$[-\pi,\pi]$上满足狄立克莱定理条件,所以$f(x)$在$[-l,l]$上的傅立叶级数$\dfrac{a_0}{2}+\displaystyle\sum_{n=1}^\infty(a_n\cos\dfrac{n\pi}{l}x+b_n\sin\dfrac{n\pi}{l}x)$收敛,并且,若其和函数为$S(x)$,则有

(1)在$f(x)$的连续点处,$S(x)=f(x)$;
\vspace{2mm}

(2)在$f(x)$的间断点$x$处,$S(x)=\dfrac{f(x-0)+f(x+0)}{2}$.
\vspace{2mm}

(3)在端点$x=\pm l$处,$S(x)=\dfrac{f(-l+0)+f(l-0)}{2}$\\\vspace{2mm}
(或在$x=0,2l$处,$S(x)=\dfrac{f(0+0)+f(2l-0)}{2}$)

其中
\begin{align*}
    & a_n=\dfrac{1}{l}\int_{-l}^l f(x)\cos\dfrac{n\pi}{l}x\deriv x, \quad n=0,1,2,\cdots \\
    & b_n=\dfrac{1}{l}\int_{-l}^l f(x)\sin\dfrac{n\pi}{l}x\deriv x, \quad n=1,2,3,\cdots
\end{align*}

可见,任意区间$[-l,l]$上的傅立叶级数是区间$[-\pi,\pi]$上的傅立叶级数的推广,而区间$[-\pi,\pi]$上的傅立叶级数是区间$[-l,l]$上的傅立叶级数的特殊情况.